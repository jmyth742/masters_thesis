\documentclass{article}
\usepackage[utf8]{inputenc}
\usepackage{amsmath}
\usepackage{graphicx}
\usepackage{float}
\usepackage{amssymb}
\usepackage{amsthm}
\usepackage[acronym]{glossaries}
\usepackage{hyperref}
\usepackage[capitalise, noabbrev]{cleveref}


% \cref of cleveref: strip Eq. from equation references
% https://tex.stackexchange.com/questions/122174/how-to-strip-eq-from-cleverf
%\crefname{equation}{}{} % Completely delete Equation, even as name
\crefformat{equation}{(#2#1#3)}
\crefrangeformat{equation}{(#3#1#4) to~(#5#2#6)}
\crefmultiformat{equation}{(#2#1#3)}%
{ and~(#2#1#3)}{, (#2#1#3)}{ and~(#2#1#3)}

\title{Masters Thesis TBMA}
\author{Jonathan Smyth}
\date{Jan 2021}

\begin{document}

\maketitle
\tableofcontents

\section{Abbreviations}
\begin{itemize}
\item Type based multiple access - TBMA
\item Evolved Node B - eNB
\item User equipment - UE
\item Information centric networking - ICN
\item Forward error correction - FEC
\item Hybrid Automatic repeat request - HARQ
\item Precoding matrix indicator - PMI 
\item Rank Indicator 
\item Channel State information - CSI 
\item Channel quality indicator - CQI 
\item Radio network identifier - RNTI
\item Physical uplink shared channel - PUSCH
\item Physical uplink control channel - PUCCH
\item Time division multiple access - TDMA 
\item Frequency division multiple access - FDMA
\item Code division multiple access - CDMA
\item Evolved packet core - EPC
\item Independent and identically distributed - I.I.D
\item Probability mass function - PMF
\item Probability density function - PDF
\item Signal to noise ration - SNR 
\item Maximum likelihood - ML
\item Acknowledgement - ACK
\item Mean Squared Error - MSE 
\item Time division duplexing - TDD
\item Frequency division duplexing - FDD
\item Uplink Control Information - UCI
\item Downlink Control Information - DCI
\item Uplink Shared channel - UL-SCH 
\item Least significant bit - LSB
\item Radio Resource Control - RRC
\item Cyclic Prefix - CP
\end{itemize} 

\section{Abstract}

\section{Thesis Structure}
%% overall general structure of the thesis and what to expect.
The thesis is structured as follows; firstly a short introduction will be given in \cref{intro}, next the problem we are facing and context of why this approach is ideal in \cref{prob_context}. With some background motivation understood, we would move onto the current state of the art approaches for feedback in LTE and multiple access methodologies which is found in \cref{sota}. We would move then into the \cref{tbma_sec} where the system model for TBMA is presented and discussed. This will give the basis of the mathematical model which is to be understood for TBMA. Data specific models are then presented in \cref{data_models} based on potential use-cases for TBMA, this will illustrate how TBMA could be applied to different data models in the real world. Once the model is understood, the process of how we can estimate the parameters based on our model is then presented in \cref{param_estimation} and discussed, this is broken down based on a random complex channel and an identical channel \cite{tbma}. After we set our theoretic framework, some examples of real world applications of TBMA are presented in an illustrative and informative manner, showing how the different subcarriers and channel structure should look, see \cref{schemas_and_channels}

Next, we would move on to the simulations and numerical analysis of the results of TBMA, comparing different features of the channel architecture that we can exploit to tune our results and optimise the channel, see \cref{sim_results}. Next a discussion on the application to the LTE network in depth and how thee feedback could be achieved given the current LTE implementation, this is discussed and presented \cref{lte_app}. Finally, future steps and work of TBMA are presented in \cref{summary_out}. 

\section{Introduction}\label{intro}
%% in the introduction i present the overall idea of what 5G communication is, some overview on the applications of the 5g and some potential drawbacks as we move into this new age of IoT and the ever expanding number of users in a network.
\subsection{Motivation and goal of thesis}
The capabilities of the 5th generation of mobile communications promise new application in variety of
fields, where some examples are autonomous driving, immersive audio/video, Internet-of-Things and Industry 4.0. While the primary focus of the first release of 3GPP New Radio (NR) is mobile communications in general, future releases will also comprise the above mentioned vertical markets. In contrast to the current release of 3GPP NR, 3GPP LTE is already addressing the vertical markets with a reduced feature set. Consequently, the current extension of 3GPP LTE can also be seen as proof-of concept for the next releases of 3GPP NR.

Applications like autonomous driving highly depend on a reliable and efficient communication link between vehicles and road infrastructure. Therefore, broadcast and multicast signals are essential to perform efficient communication with the scarce spectrum resources. However, current methods of broadcast/multicast just offer communications without link quality information to the target group. Feedback mechanisms are generally present only for unicast transmissions. In order to enable efficient feedback mechanisms for multicast communications, information centric communication approaches, e.g. based on type based multiple access (TBMA)\cite{tbma}, promise an efficient implementation. This is further justified if a large number of devices with strict latency requirements are taken into account, rendering dedicated feedback channels ineffective. If we consider an overview on how CQI feedback mechanisms work currently in the LTE framework, it is done on the PUSCH or PUCCH (Physical Up-link shared channel, Physical uplink control channel) depending on the exact settings of the transmission \cite{ETSITS136213}. We see that each user connected to an eNB is given an RNTI (Radio Network Identifier), this RNTI is used to scramble the data transmitted and to decode the received data on the eNB, both for uplink and downlink. This means that in the current setup in LTE, the eNB must decode the total number of users in a group one by one, on their specific RNTI. When considering this for a large number of grouped users, we notice that the base station must individually decode each users response based on their RNTI. This can cause bottle necking in safety critical systems, as the computation for one feedback is multiplied by the entire group size. Thus this leads to potential high level of latency in feedback from the group.

In this thesis, an information centric feedback mechanism for multicast signals shall be investigated. In this context, based on the current standard of 3GPP LTE \cite{3gpp36321} and NR\cite{3gpp38321}, the feedback mechanism of the unicast shall be extended to allow efficient multicast feedback.

The TBMA approach for feedback mechanism shall be investigated and optimised to understand what are the key performance indicators to allow for successful decoding of the information sent along the channel. In particular, the number of users, adaptive power, style of encoding the channel and the amount of re-transmissions of the data with play a centric role in deciding what is best. Upon understanding the aforementioned parameters a proof of concept practical implementation would be presented using srsLTE \cite{tbma}. 


\subsection{Problem and Context}\label{prob_context}
%% here is the section where i give the problem some context about the problem at hand, why it is important and how it can be solved.
In classical communication, the source and the channel are encoded separately, thus the channel is independent of the data \cite{shannon_theory}. However information centric communication (ICC) approaches the problem differently, by jointly encoding both the the source and the channel. ICC utilises the knowledge of both the source and the channel to best optimise the channel coding. \cite{information_centric}.
Many different approaches exist for the classical communication approach, CDMA, TDMA and FDMA [Find cite]. These approaches which have their respective pros and cons have one common drawback which inhibits them from being the ideal approach to the problem which is being addressed, they are host centric approaches. The question for this thesis is how can the design of the MAC best suits the an information centric approach. How can the MAC be designed jointly for encoding both the source and channel. 
If we consider the TBMA approach, on a high level view, we can see that to access the channel there are no strict requirements other than encoding their respective data and transmitting. If we consider this situation as is, we would think that the if all users transmit their signal at roughly the same time, the individual decomposition of the signal is difficult.   interference occurs when the maxima of two waves are 180 degrees out of phase: a positive displacement of one wave is cancelled exactly by a negative displacement of the other wave. The amplitude of the resulting wave is zero, this effect can be seen in \cref{fig:wave_inter}. In a classic approach TBMA could be susceptible to destructive interference due to having no constraints on time of transmission, however as we are interested in the collective consensus of the signal, this potential interference plays in the favour of TBMA. 

\begin{figure}
    \centering
    \includegraphics[
      width=6cm,
     height=7cm,
    keepaspectratio,]
    {interence_waves.png}
    \caption{Constructive and Destructive wave interference [http://hydrogen.physik.uni-wuppertal.de/hyperphysics/hyperphysics/hbase/sound/interf.html] }
    \label{fig:wave_inter}
\end{figure}

Consider two similar scenarios in a multicast transmission from a base station, the base station transmits some information to all its grouped users. First the eNB wants feedback information about, for example, the quality of the downlink channel . As TBMA is information centric, each encoded subcarrier on transmission has a particular non arbitrary meaning, which in the case of host-centric may not be true. As we jointly encode the source and the channel, potential interference should be mitigated which could be introduced in host-centric approaches. Second, the base station instead of feedback information, wants normal uplink traffic, which could be any kind of arbitrary data from a user, in this case, more conventional host-centric approaches may be of better use. In this thesis we show that significant gains can be realised in estimation quality and in system resource consumption if the physical layer and the multiple access are designed jointly for the purpose of estimation, not just explicit value retrieval.

Some of the main problems when approaching these larger scale multicast feedback mechanisms is how best to use the wireless channel to maximise the resource available. Usually base stations have finite resources for computation and poorly designed systems can cause latency in transmission. An example could be a huge number of cars in a city grid, a multicast transmission of information is sent to all users in the group.  In this use case, feedback using normal methods such as unicast, would consume a large portion of the resources available because the current approach mandates each feedback message from each user would be computed individually at the base station, this means the total time taken for all feedback messages to be computed and understood is the number of users in the group multiplied by the computation time of one. This approach is not feasible for huge numbers of users.

Using information centric approaches such as TBMA, would allow simultaneous usage of the resources available and substantially reduce the latency of the feedback. Users in a massive sensor network would allow for different types of information to be fed back to the base station. In LTE different modulation schemes exist which are used depending on the quality of the wireless channel. Modulations scheme play a key role in adapting to noisy environments and allow for consistent and efficient transmission of data. One such indicator which plays a role in the type of modulation used in data transmission is the Channel Quality Indicator (CQI). Consider the scenario when multiple users receiving unicast transmissions, each UE, could be anywhere in the cell region with unknown obstacles or interference, this means that there exists the situation when different modulation schemes are being used for different UEs attached to the base station. 

In the case of multicast user groups, it is not possible to modulate a single multi cast in different schemes, instead a single scheme is needed for a single multicast to many users, this scenario in itself elicits the same logically inefficient error we see in the aforementioned paragraph as all users in the group would need to relay their own CQI. Instead, a better approach to understand the collective consensus of the CQI and thus deriving the bare minimum modulation needed for a sufficient transmission would be via a TBMA approach where each user constructs a PMF of their CQI history for a given time period, this would allow the eNB to better and more efficiently adapt their multicast transmission to best suit the users. If we look at \cref{fig:pmf_subcarriers}, we can see that how we could understand a collective TBMA transmission, with the source and channel encoded with a particular goal in mind. This example of an ACK/NACK transmission could be exactly what is needed for a base station to understand an overall idea of how the multicast channel is performing. 

\begin{figure}
    \centering
    \includegraphics[
      width=8cm,
     height=12cm,
    keepaspectratio,]{cqipmf_subcarrier.png}
    \caption{PMF of all users results of feedback across available subcarriers.}
    \label{fig:pmf_subcarriers}
\end{figure}

Given the above example, we can see how it is no longer important to understand what every individuals exact measurement is, rather it is more important to get an overall understanding of how good the channel quality is for the majority of the users in a group. This can allow sensible choices of modulation schemes by the base stations and better use of radio resources as an overall, whilst satisfying the overall needs of a group instead of the individualistic. 

\section{State of the art} \label{sota}
\subsection{Multiple Access Methods}
%% standard SoTA, here i compare the  different access technologies typically found in wireless mediums, showing the positive points and negative points of each. then finally a short summary on why they are not good for huge networks with given constraints. 
This section gives an overview of the current state of the art approaches for multi access channel methodologies and how feedback mechanisms currently work in the LTE framework. A problem which is obvious in wireless communication, is how to share a common medium. A radio resource can be thought of as a proportion (this can be multiplexed by time, frequency or code) of the wireless radio medium which is available \cite{access_tech}. 

Communication channels need to be designed to meet certain requirements and with the ability to handle many users or increased efficiency of limited resources. Firstly, looking at multiple access method for the wireless channel which are host-centric approaches, currently there exists three primary methods of multiple access (there are many variations on these). Time-division, Frequency-division, Code-division are to be outlined next with their respective advantages and disadvantages. Lastly we will examine how the feedback mechanisms work currently in the LTE framework, firstly with unicast transmission and its respective aspects and finally with the current multicast (eMBMS) approach. \newline

\subsubsection{TDMA} 
Time Division multiple access,  relies on signals which have been digitised and is based on time division multiplexing \cite{tdma_info}. TDMA in essence allocates different time slots for the users of a channel to transmit their data. In TDMA systems, the spectrum of the available radio is divided into user specific time slots, only the allocated user can transmit on that respective slot \cite{tdma_info}. In \cref{fig:tdma_figure}, the cyclical style of the TDMA can be seen, each user takes their turn gaining access to the channel. TDMA uses preambles to provide identification and incidental information used at the receiver \cite{tdma_info}.
Some of the advantages and disadvantages of this approach can be seen below.\newline
Advantages:\
\begin{itemize}
    \item TDMA can easily be adapted to different transmission rates.
    \item TDMA separates user by time, so there is no need worry of interference from simultaneous transmission.
\end{itemize}
Disadvantages\newline
\begin{itemize}
    \item In TDMA, each user has a predefined time slot. Roaming from one cell to another could lead to a situation where no time-slots are available and the user is without connection.
    \item Demands high peak power on the uplink in transient mode.
\end{itemize}

\begin{figure}[H]
    \centering
    \includegraphics[
      width=6cm,
     height=7cm,
    keepaspectratio,]
    {tdma_view.png}
    \caption{Time-Division Multiple Access (FDMA) \cite{tdma_info}}
    \label{fig:tdma_figure}
\end{figure}

\subsubsection{CDMA}
Code division multiple access uses spread spectrum techniques, meaning each user is not allocated a particular frequency, instead they utilise the full available spectrum. CDMA transmissions are encoded with pseudo-random digital sequences\cite{cdma_info}.\ The receiver knows the user specific code sequence and uses said sequence to decode the received signal, this is possible because the cross correlation between different users is small. This can be achieved as the bandwidth of the code signal is chosen to be vastly larger than the information signal, the encoding spreads the signal across the entire spectrum \cite{cdma_info}. \newline
Advantages:\
\begin{itemize}
    \item CDMA provides high spectral capacity and can accommodate many users per MHz.
    \item The dropout rate only occurs when the user is at least twice as far away from the base station.
\end{itemize}
Disadvantage:\
\begin{itemize}
    \item Channel pollution, where signals from too many cell sites are present in the subscriber. s phone but none of them is dominant. When this situation arises, the quality of the audio degrades
\end{itemize}
\subsubsection{FDMA} 
Frequency division multiple access (FDMA) assigns users an individual frequency band of the entire available wireless channel frequency, this can be viewed in Figure \ref{fig:fdma_figure}. Each users' channel is allocated on demand when entering the cell region \cite[Section...]{fdma_info}. All active adjacent frequencies in a cell are divided by a guard band to reduce cross talk between channels. FDMA is based on frequency division multiplexing and each user has a separate frequency for both uplink and downlink \cite{fdma_info}.  
Advantages:\
\begin{itemize}
    \item FDMA systems have lower complexity than TDMA or CDMA.
    \item FDMA is technically simple to implement. 
\end{itemize}
Disadvantage:\
\begin{itemize}
    \item FDMA only yields modest capacity improvement from a spectral allocation.
    \item FDMA wastes bandwidth, if an FDMA channel is not in use then it cannot be used by another user in the cell. 
\end{itemize}

\begin{figure}[H]
    \centering
    \includegraphics[
      width=6cm,
     height=7cm,
    keepaspectratio,]
    {fdma_view.png}
    \caption{Frequency-Division Multiple Access (FDMA) \cite{fdma_info}}
    \label{fig:fdma_figure}
\end{figure}

\subsubsection{OFDMA}
Orthogonal frequency division multiple access (OFDMA) is the multi user variant of orthogonal frequency division multiplexing (OFDM), where multiple access is achieved by assigning different sub carriers to different users. This approach allows simultaneous data transmission from several users. The OFDMA approach means the radio resource is two-dimensional (2D), contiguous or non contiguous sub carriers span the frequency and at the same time an integer number of OFDM symbols span the time domain\cite{ofdma_info}.

Although TBMA is also an orthogonal access scheme, there exists a fundamental difference between TBMA and the aforementioned schemes. TBMA is goal orientated, meaning both the channel and source are jointly encoded with this purpose in mind. If we compare this to the other approaches, we see that these schemes allow user to transmit exactly their payload with very high degrees of accuracy with no explicit goal in mind. This is a very useful and robust approach if we are concerned about arbitrary data like we would see in everyday use of LTE.  The drawbacks as discussed due to the architectural nature of these approaches require individual usage of the channel where each user regardless of any potentially common goals in transmission are mitigated. When we know for example, all users in a group are interested in a common goal orientated transmission like a consensus on the ACK/NACK of a message, then it is redundant not to use this non arbitrary information in favour of more efficient usage of the resources available. The reason this is redundant is such that huge numbers of users in a given network could lead to serious latency which is not practical for some applications in the wild. When a network requires generalist overall distribution of the group of users then an information centric approach such as TBMA may be more suitable for time or load constrained networks.

\subsection{LTE feedback methods}\label{lte_feedback_current}

In the current LTE framework, both uplink and downlink transmission are possible. Multicast transmission is downlink traffic data which is disseminated from the eNB to the UE(s), feedback from the multicast transmission is uplink traffic. However, as discussed in the previous sections, we can see there is no dedicated feedback mechanism for multicast transmission. In LTE the eNB is interested in a plethora of metrics which are beneficial for increasing the efficiency of servicing the UEs. LTE has many different coding schemes which are used based on the quality of the channel which is influenced by interference from other cells and noise.  Link adaptation is a crucial aspect of LTE that allows for the most efficient and robust transmission of data \cite{umts_sesia}[Section 10.2]. LTE architecture has many layers for different purposes, in the case of this thesis and feedback, we are most interested in the Physical layer and the MAC layer. Looking at the specification from 3GPP we can see how the current breakdown of feedback mechanisms look. Currently, there exists both periodic and aperiodic feedback from the UE to the eNB, these types of feedback can either be on the PUCCH or the PUSCH, depending on the type of feedback requested and the configuration \cite{ETSITS136213}[Section 7.2]. To understand the feedback mechanism in better detail, an overview of the propagation of the channel structure is needed. The mapping of these channels to the physical layer from the logical layer can be viewed in the Figure \ref{fig:lte_chan_prop}.

\begin{figure}[h]
    \centering
    \includegraphics[
      width=8cm,
     height=10cm,
    keepaspectratio,]{uplink_scheme.jpg}
    \caption{LTE channel propagation}
    \label{fig:lte_chan_prop}
\end{figure}

The data initially enters the logical channel, which provide services for the Medium Access Control (MAC) layer within the LTE protocol structure. Next the data is then multiplexed to the transport channel which offers information for the MAC and higher layers, via the UL-SCH. Finally the data reaches the physical channel. 

\begin{itemize}
    \item PUCCH - used for control data, this data is transmitted in the form of Up-link control information. This usually contains information related to Scheduling request, HARQ ACK/NACK and CQI. 
    \item PUSCH - this is the main user data channel, in certain conditions it also carriers the UCI and feedback metrics ACK/NACK and CQI.
\end{itemize}

\subsubsection{Channel State indicators}
At a high level we can group these feedback items as channel state information (CSI) , which can be broken down into;
CQI - channel quality indicator, this metric gives insight into the quality of the channel at any given time. The value of CQI can be between 1 and 15, or if the CQI index if 0 then the index of 1 does not satisfy the condition that the transport block error probability has not exceeded 0.1 \cite{ETSITS136213}[Section 7.2.3]. The CQI is then used to decide which modulation scheme to use and the relevant transport block size, depending on the quality of the transmission \cite{ETSITS136213}[Section 7.2]. 

Dependant on the configuration of the UE the aforementioned CQI feedback is realised through different architectural approaches. Both time and frequency resources and configurations are chosen by the eNB in the serving cell region, these mechanisms can be either periodic or aperiodic \cite{ETSITS136213}[Section 7.2]. In the event of the both a periodic and aperiodic report being transmitted in the same sub frame, the aperiodic report would be only transmitted. 

The respective channels for aperiodic and periodic transmission are shown below;
\begin{center}
 \begin{tabular}{||c c c||} 
 \hline
 Scheduling mode & Periodic CQI & Aperiodic CQI  \\ [0.1ex] 
 \hline\hline
 Frequency non-selective & PUCCH &  \\ 
 \hline
 Frequency selective  & PUCCH & PUSCH\\
 \hline
\end{tabular}
\end{center}
% where frequency selective is defined in general as, 
%% should i mention more indepth infomration here i.e...is done on the $n +k$ subframe of uplink?
The aperiodic reporting of CQI is done on the PUSCH can be triggered by the eNB through setting the CQI request bit in an uplink resource grant sent on the PDCCH . The exact type of the CQI is configured on the eNB through the RRC signalling. CQI reporting has a few distinct types \cite{umts_sesia}[Section 10.2.1.1]. 
\begin{itemize}
    \item Wideband feedback - UE reports a CQI value for the whole system bandwidth available.
    \item eNB configured Sub-band Feedback - UE reports the Wideband report, additionally a CQI value for each sub-band. Sub-band CQI reports are encoded deferentially with respect to the wideband.
    \item UE selected sub-band feedback - The UE selects a set of preferred sub-bands of a predefined size within the entire bandwidth.  The UE reports one wideband CQI with one CQI report reflecting the average quality of the predefined size. The UE will also report the position of said sub-bands.
    
\end{itemize}

If the respective CQI field is not set reserved for other purposes then the appropriate field bit can be toggled. The combinations of the triggering mechanism for this report can be viewed below \cite{ETSITS136213}[Section 7.2.1]: 

\begin{center}
 \begin{tabular}{||c c||} 
 \hline
  CSI request field & Description \\ [0.1ex] 
 \hline\hline
 '00' & No aperiodic trigger  \\ 
 \hline
 '01'  & Aperiodic report for serving cell\\
 \hline
 '10' & Aperiodic for 1st set of serving cells\\ 
 \hline
 '11'  & Aperiodic for 2nd set of serving cells\\
 \hline
\end{tabular}
\end{center}

These respective trigger values are decoded by the UE on the PDCCH with an uplink DCI format in the UE specific search space \cite{ETSITS136213}[Section 7.2.1, Table 7.2].
It is worth noting that there exists a minimal reporting interval for any aperiodic transmission, which is 1 sub frame. Furthermore, when aperiodic feedback occurs with no associated transport block of user data, the UE shall utilise the PUCCH instead of the PUSCH.

%% the 3gpp has a lot of configurational approaches for when and how the CSI is transmitted ....should i go into detail about all of these??

As discussed earlier, periodic reporting of the CQI also occurs in LTE. UEs are semi-statically configured by the higher layers for periodic reporting of such metrics, which are present on the PUCCH apposed to the PUSCH \cite[Section 7.2.2]{ETSITS136213}. Like most features in the LTE framework, there exists an abundance of possible configurations. Like Aperiodic reporting, periodic reporting supports both wideband and ue-selected sub-band, however eNB-selected is not supported. Similarly, the type of periodic reporting is configured by the eNB higher signalling layers, RRC \cite{umts_sesia}[Section 10.2.1.2]. Periodic wideband feedback is similar to that of the aperiodic which is sent along the PUSCH, however the ue-selected sub-band is different. In UE-selected sub-band, the total number of sub-bands $N$ is divided into $J$ fractions called bandwidth parts. The value of $J$ is dependant on the system bandwidth, in this case, one CQI value is computer and reported for a single sub-band from each bandwidth part, along with its respective sub-band index.

PMI - Precoding matrix indicator is used to determine how the individual data streams (layers in LTE) are mapped to the antennas on the UE. Carefully selecting this matrix yields the maximum number of data bits that the UE can receive on across all layers. If the UE knows what the allowed precoding matrix are, then the UE can send a PMI report to suggest to the eNB the most suitable matrix to use \cite{csi_defs}.

RI - Rank indicator,  this is the number of layers and number of different data streams transmitted in the downlink. The aim of an optimised RI is to maximise the channel capacity across the entire available bandwidth by taking advantage of the each full channel rank  \cite{csi_defs}. 

\subsubsection{HARQ ACK/NACK}

Additionally the LTE framework has another form of feedback that is also transmitted on the PUCCH. The Hybrid Automatic Repeat Request (HARQ) is also utilised. The HARQ consists of two parts, firstly the ARQ in LTE is a mechanism that as the name suggests repeats the request if an acknowledgement (ACK) is not received by the sender after a predefined timeout period.  After this timeout period the receiver discards the bad packets and the sender will re-transmit. The next step is the Forward Error Correction (FEC), this process takes the malformed data packets and stores them in a buffer until the next transmission, the idea is that 2 or more packets received with insufficient information to decode them alone can be combined together in such a way they can produce a signal that can be decoded \cite{3gpp36321}[5.3.2 HARQ operation].

The HARQ mechanism in LTE is encoded into the UCI and carried usually on the PUCCH, however dependant on some conditions can also be carried on the PUSCH. The UCI format dictates the type of feedback metric to be transmitted, which we can see in the \cref{tab:pucch_formats} \cite[Section 17.3.1.2]{umts_sesia}. 

\begin{table}[h]
\centering
\begin{tabular}{||c| c||} 
\hline
PUCCH Format & Uplink Control Information (UCI) \\ [0.1ex] 
\hline\hline
Format 1 & Scheduling request (SR) (unmodulated waveform)\\ 
\hline
Format 1a & 1-bit HARQ ACK/NACK with/without SR\\
\hline
Format 1b & 2-bit HARQ ACK/NACK with/without SR\\
\hline
Format 2 &  CQI (20 coded bits)\\
\hline
Format 2 & CQI and 1- or 2-bit HARQ ACK/NACK (20 bits) for extended CP only\\
\hline
Format 2a & CQI and 1-bit HARQ ACK/NACK (20 + 1 coded bits)\\
\hline
Format 2b & CQI and 2-bit HARQ ACK/NACK (20 + 2 coded bits)\\
\hline
\end{tabular}
\caption{PUCCH formats and their properties}
\label{tab:pucch_formats}
\end{table}


%% sufficient to use the coding scheeme not further explaination. the drawbacks being the coherent - use pilots , and a lot of redunant resources being used , 12 bits of information being sent. 
The HARQ ACK/NACK procedure occurs when a the UE receives some user data on the PDSCH. The UE will then perform its integrity check on the received data and depending on the current setting will trigger a ACK/NACK response to the received payload. Upon receiving a payload from the eNB, the UE can send its ACK/NACK response on either the PUSCH or PUCCH, if there is no user traffic in the uplink then the UE will transmit its response on the PUCCH, however, if there is user traffic (this is visible when there is a valid uplink grant from the UE), the response is encoded into the PUSCH \cite[Section 11.4]{lte_advaned_mobile}. The procedures flow can be viewed in \cref{fig:ack_nack}. 

\begin{figure}[h]
    \centering
    \includegraphics[
      width=8cm,
     height=10cm,
    keepaspectratio,]{ack_proc.jpg}
    \caption{LTE procedure for ACK/NACK response.}
    \label{fig:ack_nack}
\end{figure}

When the PUCCH is being utilised to transmit the feedback control data, a region of 2 resource blocks is assigned for the UE. The two resource blocks are mirrored and placed at either sides of the bandwidth to provide some integrity and spread diversity in the signal \cite[Section 11.4.1]{lte_advaned_mobile}. However, each UE having 2 resource blocks during on sub frame is much to large and not very good utilisation of resources, instead these resource blocks are shared amongst multiple UEs in a serving cell.  We can see from \cref{fig:pucchformats} that each slot (half of a sub frame), has the relevant uplink information doubly encoded at either end. If we consider the feedback for a given UE/set of UEs is transmitted we would see for example that same information is present in both cases of the $m=2$ in the \cref{fig:pucchformats}. Multiple UEs share the same regions of a sub frame mentioned before, 


\begin{figure}[h]
    \centering
    \includegraphics[
      width=6cm,
     height=8cm,
    keepaspectratio,]{pucch_formats_rbs.png}
    \caption{Resource Assignment PUCCH Format 1/2}
    \label{fig:pucchformats}
\end{figure}



As we see in \cref{tab:pucch_formats} is possible to multiplex both the CQI and HARQ ACK/NACK data from a UE to the eNB on the PUCCH, providing the higher layers have not disabled this feature, when this is disabled the UE may only transmit the HARQ. However, in sub frames where the eNB scheduler has allowed for simultaneous transmission of both metrics a multiplex of both aspects needs to be achieved. In this case there are two situations of note, for normal CP and extended CP \cite{umts_sesia}[Section 17.3.3]. 

Looking first at the multiplexing of CQI and HARQ with normal cyclic prefix (DCI format 2a/2b). To transmit a 1 or 2 bit HARQ ACK/NACK together with the CQI, the ACK/NACK bits are not scrambled, but only modulated with either BPSK/QPSK, resulting in a single modulation symbol $d_{HARQ}$.  This single modulation symbol $d_{HARQ}$ is then used to modulated the second Reference Signal (RS) symbol in each CQI slot, meaning the ACK/NACK is actually signalled through the RS itself.

%% should i add in here more information about the super specifics about subframe allocation etc???




\section{Type Based Multiple Access}\label{tbma_sec}
In this section, we will go in detail the the systems model for TBMA to be realised. Next, potential data models are then presented and discussed. Thirdly, how the parameter estimation of the TBMA is presented for both identical and random channels. Finally, potential LTE channel schemes are presented and discussed with applications within the LTE framework. 
\subsection{System Model}\label{sys_mod}
% here the system model is presented, with some prior constraints and assumptions about the approach. 
% it is divided into the channel h=1 and random, then the following data-models are presented and discussed based on the model, some observations of the data are presented
We will present and discuss the system model for TBMA. To begin, we should first look from a high level what exactly is happening. In a simplistic manner, TBMA is a method of transmitting information when the source and channel are jointly designed. To easily understand this concept we should first look at the likes of host-centric approaches, where the channel and source not jointly designed. In Host-centric systems, the channel is designed to handle data transfer regardless of the source it has come from. Likewise the source is designed irrespective of how the channel looks. Host-centric approaches may be beneficial for arbitrary data when the information of each specific device is of interest. However, in the case of TBMA, we may not actually be interested in the specific information each user has, rather combined information that can yield metric of interest, such as PMFs. 

In layman's terms, TBMA can be viewed up as an approach to transmit data when the sender and receiver adhere to a coding scheme on both the channel and the sources. If we take a given signal $s$, that each user, $k$ in the group will transmit along a channel $h$. When $s$ and $h$ are jointly designed with the design known to all actors in the environment. Given this rhetoric, we can see that for each of the given signals $s$ convoluted with the channel $h$, then the final outcome of this type of transmission can be seen as $\sum h \cdot s$. As discussed, this is a simplified view of what is happening. Next we discuss a more accurate representation of the model. 


First, we  consider a set of User Equipment (UE) or nodes, where ${k}$ of ${K} = K$ which are conditionally independent and identical distributed (i.i.d). Each UE jointly observes data, $X^i_1, \ldots, X^i_K$, given a parameter $\theta$ and $i$ being the observed index. It is assumed that every observation  ${X^i_k \in \{1, \ldots, R\}}$, takes only discrete values with a probability mass function (PMF) ${p_{\theta} = [p_{\theta}(1), \ldots, p_{\theta}(R)]}$, where each $p_{\theta}$, corresponds to a certain real world value. The PMF belongs to a family, $\{p_{\theta}: \theta \in \Theta\}$ where $\Theta \subset \mathbb{R}$ is the parameter space.

% $\mathcal{K} the set of all devices: \mathcal(K)=\{1...K\}$


The local observation of each UE is given by the $X^i_k$. For each UE in the network to access the wireless channel according to the TBMA approach, a UE would transmit its available encoded subcarriers in a waveform such that, $\boldsymbol{s} \in \mathbb{C}^{N \times 1}$, where the outcome of the observation determines which subcarrier is selected (this corresponds to the jointly designed channel and source, where each available subcarrier corresponds to a design choice. i.e. subcarrier 7 =  CQI values of 7) from the orthonormal signature matrix $\boldsymbol{S} \in \mathbb{C}^{N \times R}$, below in \cref{fig:sig_mat}, we can see a simple example of a signature matrix.

\begin{figure}[ht]
\centering
    $\begin{pmatrix}
\pm1 & 0 & \hdots & 0 & 0\\
0 & \pm1 & \hdots & 0 & 0 \\
\vdots & \vdots & \ddots & \vdots & \vdots \\
0 & 0 & \hdots & \pm1 & 0 \\
0 & 0 & \hdots & 0 & \pm1 \\
\end{pmatrix}$
\caption{$n \times n$ Signature matrix example.}
\label{fig:sig_mat}
\end{figure}

Before presenting the TBMA equation in full, lets look at how the different aspect fit together. Based on the description of TBMA, we know that the idea of this approach is to jointly design the channel and source for a goal orientated purpose. Thus, when we think about the formulation of the equation we would expect there to be a resemblance of this idea.  We consider that a group of users, $K$, all want to transmit their information in the case of LTE to a common eNB. Each of $k$, has a predefined selection of subcarriers available to encode the outcome of their observation in, $S_X$, each of these users, $k$ will in turn transmit their symbols and observation $S_X$ along a wireless medium, $h$, which is modelled as being different for different users, but the same across a given $S_X$ transmission, for each user. Based on these assumptions how the definition of TBMA fits in, we can see that at the fusion centre we must sum all of these signals to attain the overall waveform, this is important, as this plays on the fact that the signal and channel are designed with a goal in mind. Additionally, due to the additive nature of a wireless medium the received signal at the $i$-th is actually observed at the fusion centre as;

\begin{align}
    \boldsymbol{y}^i = \sum_{k \in \mathcal{K}} h^i_k \boldsymbol{s}_{X^i_k} + \boldsymbol{w}^i, \label{eq:system_model}
\end{align}

where $h_k^i$ is the complex channel, which is assumed to remain constant during any given $i$-th transmission but differ between $k$. We observe the modelled noise, $w^i$ as being $\mathcal{CN}(0, \sigma^2 \boldsymbol{I}_N)$. The equation presented in \cref{eq:system_model}, outlines how each user transmits their respective payload encoded into the available subcarriers across a wireless channel medium subject to some distortion depending on the environment, this signal is then gathered, summed and finally some complex white Gaussian noise is added.


After observing a received signal, the fusion centre will apply matched filters to the orthonormal basis $\left\{ \boldsymbol{s}_r\right\}_{r \in [R]}$ which gives
%
\begin{align}
    \boldsymbol{z}^i &\triangleq \left[\langle \boldsymbol{y}^i, \boldsymbol{s}_1\rangle, \ldots, \langle \boldsymbol{y}^i, \boldsymbol{s}_R\rangle  \right]^T, \label{eq:mf_output}
\end{align}
%
where $\langle \boldsymbol{y}^i, \boldsymbol{s}_r\rangle$ denote the inner product of the received signal and $\boldsymbol{s}_r$ at the $i$-th transmission. Where the matched filter is the scalar product of $y$ with the orthonormal signature, because the signature is orthonormal, all the other information in the signal is removed and thus is matched to the desired signature we can see this effect in \cref{fig:sig_mat}. 

There exist some constraints and assumptions for the system model. It is assumed that the energy constraint $||s_{i}^2|| \leq 1$, where the signal of each transmission does not exceed more than 1 energy unit, this means that each user should not transmit their signal with any extra power. This will play an important role later on when we discuss the re-transmission of the signal. It is also assumed that all noise and channel response (when applicable) is modelled as white complex Gaussian.   
\subsection{Data Models}\label{data_models}
% in this part i talk about how the different scenarios of data could be understood. 

Looking at how to model the data, there are three distinct use cases for this section. First looking at how to estimate a single common event value, for example, each user is transmitting either $1$ or $0$, where if the common event occurs, $1$ is transmitted otherwise 0. Next, we move to a more granular approach for estimating a range of common events. Finally we take another step and look at how we could transmit and understand a PMF of real world values. 
\subsubsection{Case~I}
%% JUST ACK EVENT

In the case of this thesis, we are interesting in some applications of the TBMA approach. Consider a set of $K$ sensor nodes, measuring a common event where, ,${X \in \{1, 0\}}$ ${X \in \{1, \ldots, R\}}$. Nodes in the system can be prone to errors or may be out of scope of the event, thus we have a measurement vector of ${x \in \{0, 1\}}$ which is obtained by the following :

\begin{align}
    \boldsymbol{x}_{[r]} 
    = 
    \begin{cases}
        1 & X = r
        \\[2ex]
        0 & \text{ else}.
    \end{cases}
    . \label{eq:event01common}
\end{align}

In the end, the overall idea of the fusion centre is to obtain the most likely event that occurs. Equation \ref{eq:event01common}, depicts the scenario when all users have a binary event, all users will transmit either a 1 when the common event occurs or a 0. 

\subsubsection{Case~II}
%% CQI INDIVIDUAL
At each node one out of $R$ events occur, where the events can be correlated among all nodes or completely independent. This local event~${X_k \in \{1, \ldots, R\}}$ is measured by each node. Thus, the measurement vector $\boldsymbol{x} \in \{0,1\}^{R}$ of each sensor node in~\ref{eq:system_model} is obtained by
\begin{align}
    \boldsymbol{x}_{[r]} 
    = 
    \begin{cases}
        1 & X_k = r
        \\[2ex]
        0 & \text{ else}.
    \end{cases}
    . \label{eq:event02independent}
\end{align}
Here, the objective at the fusion centre is to obtain the PMF of all measured events by observing~\ref{eq:system_model}. In this case, each available subcarrier $s_i$ for a given user $k$ observes event $x$, where the encoded subcarrier of the event corresponds to a variety of actual values in the real world. This approach allows for different events to be observed and then transmitted.

\subsubsection{Case~III}\label{case3}
%%% CQI PMF EVENT
At each node a PMF~$p_{\theta,k}$ is measured, where the PMF among all nodes can be correlated or independent. Thus, the measurement vector $\boldsymbol{x} \in \mathcal{R}^{R}$ of each sensor node in~\ref{eq:system_model} is obtained by
\begin{align}
    \boldsymbol{x} = p_{\theta,k}
    , \label{eq:event03PMF}
\end{align}
Here, as in case~II, the objective at the fusion centre is to obtain the PMF of all measured events by observing~\ref{eq:system_model}. Since each user $k$, already observes and transmits a PMF, the underlying distribution is already well better approximated at the sensor, due to the fact that the received signal at the fusion centre already inherently has the underlying PMF from multiple different sources. Thus, we assume that the performance at the receiver (e.g. the mean squared error) is better than if a single measurement is transmitted like in case~II, this assumption would hold true as more information about any given environment and situation is known to reduce the degree of uncertainty in the measurement.

Case I, II and III will be illustrated again in Section \ref{schemas_and_channels}, where they will be presented in the form of a real world example. 

\subsection{Parameter Estimation}\label{param_estimation}
%% here i talk about how to estimate the parameter or pmf of the transmitted signal, again i split this into both h=1 and random. 

As presented in the previous section, we are interested in three different data-model use cases. The idea of a TBMA  approach is to best design the channel so that the error of the estimated parameter can be minimised, such that it tends toward 0.  In some cases, we may want to determine an estimate of the actual parameter, $\theta$, in other cases we are interested in the underlying PMF, $p_{\theta}$ of the data transmitted. Respectively they are formally defined as:

\begin{align}
\hat{\theta}^* = \min_{\hat{\theta}} D(\hat{\theta}\|{\theta}),      
\end{align}

\begin{align}
\hat{p}^* = \min_{\hat{p}} D(\hat{p}\|p_{\theta}),  
\end{align}

Where, $\hat{\theta}$ and $\hat{p}^*$ are the parameter and PMF of the received signal respectively which exist in the defined probability space $[R]$.  We assume that there is no prior knowledge of the parameter $\theta$ or $p_{\theta}$, since the idea of the approach is to derive an estimate of the distribution of the data.


\subsubsection{Deterministic Channel}

It is assumed that the data is i.i.d between sensors but constant during different observation periods. For the specific case of identical channels ${(h^i_k = 1, \forall k)}$, the output of the bank of matched filters in Equation (\ref{eq:mf_output}) contains a noisy version of the histogram, scaled by the number of users, written as
%
\begin{align}
    \boldsymbol{z}^i &= K \tilde{\boldsymbol{p}}^i + \boldsymbol{v}^i, 
\end{align}
%
where $\tilde{\boldsymbol{p}}^i = \frac{1}{K}[N_1^i, \ldots, N_R^i]$ is the \emph{empirical measure} or \emph{type} with
\begin{align}
    N_j^i = \sum_{k \in \mathcal{K}}\mathbb{1}(X^i_k = j)
\end{align}
and $\boldsymbol{v}^i \sim \mathcal{CN}(0, {\sigma^2 \boldsymbol{I}_R})$, which is the complex Gaussian noise.
%
Therefore in the case of this deterministic channel, the estimate can directly obtain the histogram from the $i$-th measurement as:
%
\begin{align}
    \hat{p}^i &\approx \boldsymbol{z}^i / K. \label{eq:empf_h_const}
\end{align}

Given the above estimation method, we observe some notable pieces of information, for $K \rightarrow \infty$: The estimate (\ref{eq:empf_h_const}) converges to the true PMF in the region of high SNR values where the noise is playing less of an influential role on the measurement, i.e.~$D(\hat{p}\| p_{\theta}) \rightarrow 0$. Conversely, with low SNR values, we can improve the estimator can be by averaging over multiple observation times $i$, this approach could be employed when SNR is low, however a trade of may be the additional re-sampling. 

For $K <\infty:$ The estimated PMF (\ref{eq:empf_h_const}) describes the current sample but ~$D(\hat{p}\| p_{\theta}) \neq 0$. With re-sampling over multiple transmission intervals (e.g. by selecting random subsets $\mathcal{K}^i \subset \mathcal{K}$ for transmission during the $i$-th observation interval) different measures of accuracy can be assigned (bias, variance or prediction error) to the sample estimates.


\subsubsection{Random Complex Channel}

In the real world, it is more often than not less likely that a transmitted and receiver are in a direct line of sight (direct line of sight is when the transmission is unobstructed or reflected by obstacles in the environment). What tends to happen is that in busy environments there are obstacles such as buildings and walls that will obstruct the path of the disseminating wireless waves. The situation leads to multi path waves arriving at the receiver at different time intervals and power due to the physical affect the aforementioned obstacles have on the physicality of the waves.[ref]

This means that slightly varying waveform of the original signal are received, these can be out of phase and can be viewed from an $x$ and $y$ component, having both real and imaginary values, which can be summed to give a complete imaginary component. This effect can be viewed in Figure \ref{fig:rayleigh_fade_comps}, where the dotted lines are the different aspects of the signal received and the final complex component in bold. 

\begin{figure}[h]
    \centering
    \includegraphics[
      width=6cm,
     height=7 cm,
    keepaspectratio,]{rayleigh_graph.jpg}
    \caption{Rayleigh fading components of signal. }
    \label{fig:rayleigh_fade_comps}
\end{figure}


The amount of fade for each path which is received at the user is considered to be i.i.d and based on the central limit theorem, when adding up multiple random variables for each of the $x$ and $y$ components we attain a Gaussian PDF for each of the axis. If we then consider that each of these Gaussian are independent, the overall PDF can be obtained by multiplying both PDFs together. 

With channel coefficients drawn i.i.d. between sensors and observation intervals $h^i_k \sim \mathcal{CN}(0, 1)$, the output of the matched filter is a vector of random variables, where the pdf of the $j$-th element is given by $p_{\mathbf{z}}(\boldsymbol{z}^i[j]; N_j)$, with  
%
\begin{align}
    p_{\mathbf{z}}(x; \sigma) &= \frac{x}{\sigma^2} e^{\frac{-x^2}{2\sigma^2}} ; 
    \newline 0 < x < \infty
\end{align} 
%
% should i do the proof here for MLE of Rayleigh?
Where the likelihood function of is :

\begin{align}
    L(\theta) = \prod_{i=1}^n\frac{2x_{i}}{\theta^2}e^{-\frac{x_{i}^2}{\theta^2}}
\end{align}
\begin{align}
    L(\theta) = \prod_{i=1}^n x_{i}(\frac{2}{\theta^2})^n e^{-\frac{\sum{x^2}}{\theta^2}}
\end{align}

Then taking the $ln$ log of both sides to simplify:

\begin{align}
    \ln{L(\theta)}= \ln{(x_{1},x_{2}\dots,x_{n})} + n\ln{2} - n\ln\theta^2 {-\frac{\sum{x^2}}{\theta^2}}\ln{e}
\end{align}
\begin{align}
    \ln{L(\theta)}= \sum\ln{x_{i}} \cdot + n\ln{2} - 2n\ln{\theta} -\frac{\sum{x^2}}{\theta^2}
\end{align}

Differentiating with respect to $\theta$, gives:

\begin{align}
    \frac{\partial }{\partial \theta}\ln L{(\theta)} = -2n\frac{\partial}{\partial\theta}\ln\theta - \sum x_{i}^2\frac{\partial}{\partial\theta}(\frac{1}{\theta^2})
\end{align}

\begin{align}
    \frac{\partial }{\partial \theta}\ln L{(\theta)} = -\frac{2n}{\theta} + \frac{2\sum x_{i}^2}{\theta^3} \dots \dots \dots
\end{align}

Now setting the partial  derivative of $\theta$ equal to $0$ so we can get a maximum.

\begin{align}
    \frac{\partial }{\partial \theta}\ln L{(\theta)} = 0
\end{align}

\begin{align}
    -\frac{2n}{\theta} + \frac{2\sum x_{i}^2}{\theta^3} = 0
\end{align}
\begin{align}
    \frac{n}{\theta} = \frac{2\sum x_{i}^2}{\theta^3}
\end{align}
\begin{align}
   n\theta^2= \sum x_{i}^2
\end{align}

Thus we attain:

\begin{align}
    \theta^2 = \frac{\sum x_{i}^2}{n}
\end{align}


Hence, rearranging, the unbiased ML estimate of the $j$-th element of $\tilde{\boldsymbol{p}}$ can be obtained after observing $L$ transmissions as
%
\begin{align}
    \hat{p}[j] &= \sqrt{\frac{1}{2L} \sum_{i = 1}^{L}\boldsymbol{z}^i[j]}.\label{eq:empf_h_rand}
\end{align}

In other words, with a noisy, random complex channel, we can exploit the noise over multiple re transmissions of the exact same information to allow the estimator to reduce the effect of the stochastic noise allowing a more accurate reading to be obtained. 

As the deterministic channel, the complex channel channel also allows us to derive some observations. For $K \rightarrow \infty$: The estimate (\ref{eq:empf_h_rand}) converges to the true PMF as $L \rightarrow \infty$. However, this is not of practical relevance, due to the associated costs in re-transmitting the signal in such high order of magnitudes needed. For $K <\infty:$ With $L \rightarrow \infty$, the estimated ePMF~(\ref{eq:empf_h_rand}) describes the \emph{current sample} but ~$D(\hat{p}\| p_{\theta}) \neq 0$. 

\subsection{Channel Schemes and application} \label{schemas_and_channels}
The idea of how best to form the TBMA channel to maximise the likelihood of estimation of the empirical data. Different use cases exist to best utilise the channel. This next section describe potential examples of different use cases for the aforementioned approach. First, a generic event based encoding is shown. This approach illustrates how any generic event is could be encoded to the channel. Next, the data models from the previous equations, \ref{eq:event01common},\ref{eq:event02independent}, \ref{eq:event03PMF} will be given applications and illustrations.  

\begin{figure}[H]
    \centering
    \textbf{Schema: Generic event.}\par\medskip
    \includegraphics[
      width=12cm,
     height=14cm,
    keepaspectratio,]{events.jpg}
    \caption{Encoding a generic event to the TBMA Channel.}
    \label{fig:generic_events}
\end{figure}

The Schema presented in figure \ref{fig:generic_events}, shows the generic approach for taking any event and encoding it to any given available subcarrier at random. The Figure \ref{fig:generic_events} has the same notation as Figure \ref{eq:system_model} to help visually understand how the equations fit together. The same constraints for the equations apply. Any user in the network can take any available subcarrier, $S$, and encode a single bit which correspond to that of an event in the real world. 

With the basic understanding graphically how to encode the channel and how it works in simplest forms, looking next at an LTE specific use case. During a multicast, it would be beneficial for the base state (eNB) to understand if a users in a multicast group has received the message. To do so, a simple ACK response could be elicited by each user for the eNB to understand if the message was received by all user equipment (UE) and whether that information was received correctly, or not. However given current methodology using the physical downlink shared channel (PDSCH) for each user in the group, this would entail each user taking up some dedicated radio resource on a dedicated channel to relay that information to the eNB. 

Consider, a group of $1000$ users, $k$, where each user would encode one symbol for TBMA. In LTE, the transmission of one sub frame, takes $0.5ms$, **need to calculate the latency of huge user groups etc. 

For large networks, this would be a time consuming, resource intensive task and is it not efficient. However, using the likes of TBMA could mitigate these resource issues, each user would transmit the ACK (high bit) on a random or dedicated subcarrier, in the TBMA channel range. 
% The eNB in turn could extrapolate the amount of users in a network which corresponds to this collective signal that is transmitted on the TBMA channel. 

\begin{figure}[H]
    \centering
    \textbf{Schema: Encoded ACK}\par\medskip
    \includegraphics[
      width=12cm,
     height=14cm,
    keepaspectratio,]{acks_png.jpg}
    \caption{Random allocation of ACKs on all subcarriers.}
    \label{fig:ack_schema}
\end{figure}


Figure \ref{fig:ack_schema}, shows how the above process could work. By spreading each users ACK signal across the available subcarrier range, we could introduce signal diversity, where the effect of noise is not concentrated on a specific subcarrier, rather this is spread out across all available.
This would reduce the the weighting of noise when the normalisation of the signal happens, giving us a more realistic estimate. 
% Consider a user $k$ transmits the above scheme with 2 available subcarriers. When the user transmits $S_{X_{i}}$, and the ACK signal is encoded to the $0-th$ element of the subcarrier matrix and the remaining is left empty . We would observe that signal plus noise
% \begin{align}
%     S_{X_{0}} = 1 + w
% \end{align}

% \begin{align}
%     S_{X_{1}} = 0 + w
% \end{align}

% in this case, when we normalise 

In the LTE framework, this approach could be used for the base station to understand if the payload was successfully received by the UE. This application is comparable to that of the data model in Case I, where a common event is shared amongst all users.


\begin{figure}[H]
    \centering
    \textbf{Schema: Encoded Halves, Each User XOR ACK-NACK}\par\medskip
    \includegraphics[
      width=12cm,
     height=14cm,
    keepaspectratio,]{ack_halves.jpg}
    \caption{Halved One shot random XOR allocation of subcarriers, from 1 user to N. }
    \label{fig:xor_schema}
\end{figure}

The schema depicted in \ref{fig:ack_schema}, has some drawbacks. If we do not know the size of the multicast group then it would be hard to gauge how many users have successfully received the intended transmission. For this reason, an additional architectural aspect could be introduced. Splitting the available subcarriers into two halves could allow for a more granular understanding of the current state of the system, having a situation where comparative values are available allows for some interesting information to be extracted. 

If we take the example of autonomous cars in a smart city, where there are a lot of users, a lot of potential interference and reflection from buildings and other wireless communications, in this kind of environment, it may be understandable that some data received is corrupt and incomplete. However, having two different pieces of information available, like ACK/NACK, would allow for understanding on data corruption ratios and potential numbers of users. 

For example, if there are an unknown number of cars $C$ in a cell region and the fusion centre receives signal, $Z$ of the TBMA channel using Figure \ref{fig:xor_schema}, then the base station could understand a ratio of the integrity of the received data for all of the users, consider;

\begin{align}
    Ratio_{ACKs} = \frac{Z_{0}\ldots Z_{\frac{S}{2}}}{Z_{\frac{S}{2}}\ldots Z_{S}}
\end{align}

Where $S$ is the total number of subcarriers available in that channel construction, this approach would allow the base station to understand what proportion of the total number of users is receiving well formed data packets. Further information could be extracted if it is assumed that each user transmits with the same energy, $E$, then the base station knows what energy each user is transmitting with, thus the total energy received by the fusion centre;
\begin{align}
    Z_{K}  = K \times E_{k} + w
\end{align}
\begin{align}
    \implies K \approx \frac{Z_{K}}{E_{K}}.
\end{align}
Where the number of users can be approximated based on the total energy of the received signal, this holds true because of the energy constraint in [REF], this is an approximation because of the additive nature of noise in the transmission. 

In the LTE framework a very useful piece of feedback information each user can give the eNB is the Channel quality indicator (CQI) [REF], The CQI is, as the name suggests an indication of the quality of the channel each user is experiencing. This CQI value allows the eNB to best understand which modulation scheme to use for transmission. The CQI value ranges from 0-15 (modulation from QPSK to 64QAM), with 0 meaning the UE is out of cell range and 15 being the best case scenario. This is a very useful tool for the eNB to understand how efficiently depending on the external environmental factors to encode the information along the channel. 

The same situation applies here as the ACK/NACK scenario, the users should best use the channel to convey the collective census on the situation of the channel, with potentially huge amounts of users in a cell region, the feedback from multicast could be hard to handle. Using a TBMA channel with 16 subcarriers, each user could transmit an encoded bit in their respective subcarriers which corresponds to a real world value for the CQI. Case II from the data model is illustrated here, each user has a common set of different values which can be taken. 

\begin{figure}[H]
    \centering
    \textbf{Schema: CQI Individual structured encoding}\par\medskip
    \includegraphics[
      width=12cm,
     height=14cm,
    keepaspectratio,]{CQI_individual.jpg}
    \caption{Transmitted encoding of user gathered CQI information. }
    \label{fig:cqi_ind}
\end{figure}

As seen in Figure \ref{fig:cqi_ind}, each available subcarrier corresponds to a real world CQI value. This approach allows the eNB to get a snapshot of the current quality of the channel from multiple users. The eNB would receive at its fusion centre a PMF of the group of users CQI values at that instant, this would provide an overview of how all users in a group are receiving in the channel and allow the eNB to more appropriately adjust its encoding scheme to suit the majority of users. 

Finally, the proposed channel on Figure \ref{fig:cqi_ind} is not without fault, error in the world of measurements is something that cannot be avoided, for this reason when measuring anything it is good practice to take a set of measurement to iron out any anomalies which may have occurred. 

Therefore, a step further for the CQI which could yield more stable and realistic results for the quality of a channel over time could be each user transmitting a recorded PMF of their respective CQI over time. This approach can be viewed in Figure \ref{fig:cqi_pmf}. This approach might be more realistic than its counterpart in the previous paragraph, due to the fluctuating nature of CQI in an outdoor environment. With the eNB having a more generalist snapshot of the CQI overtime vs an instantaneous approach could avoid unnecessary modulation coding changes when they are not needed. 

\begin{figure}[H]
    \centering
    \textbf{Schema: CQI one shot structured encoding}\par\medskip
    \includegraphics[
      width=12cm,
     height=14cm,
    keepaspectratio,]{CQI_individual.jpg}
    \caption{Transmitted encoding of user gathered CQI information. }
    \label{fig:cqi_pmf}
\end{figure}

It is worth noting that in the case of this approach, that the total number of energy of each transmission is constant. Therefore, each users PMF of measurements for a given $i-th$ transmission must satisfy:
\begin{align}
    E = \sum \boldsymbol{s}_{X^i_k} 
\end{align}

Such that if a user $k$, transmits a PMF of CQI values, each encoded subcarrier must only have $\frac{1}{X}$ proportion of the total energy available for each user. This is an important consideration and assumption to allow for further analysis of the received signal. 


As different scenarios are presented, some return higher levels of granularity than others. Above, some scenarios have been presented where only general information about ratios are of interest, however the scheme presented in Figure \ref{fig:cqi_pmf}, touches on how we can deduce more accurate types of information, these encoding and transmitted channels would not only allow the base station to understand which region of CQI users are in, but also as each user transmits a PMF, potentially we could also understand the variance of their channel quality. 


\section{Simulation Results and analysis}\label{sim_results}
In this section the results from the numerical simulations will be presented with some analysis. Prior assumptions will be outlined before the the numerical simulation is presented, then a short discussion on the results comparing the assumptions with the real outcome. 
\subsection{Evaluations methods}
To evaluate the results of the simulation, two approaches were considered. First looking at the Mean Squared Error (MSE) of the ground truth and the outcome of the TBMA.  Some prior assumptions about the environment are also considered, where all simulations are done for an $SNR \in (-10\dots40)dB$, all users transmit their respective payload with the energy constraint  $||s_{i}^2|| \leq 1$.  

Where the SNR and $\sigma2$ defined as:
\begin{equation} 
SNR = 10^{\frac{SNR_{i}}{10}}    
\end{equation}
Where each SNR value is calculated at each individual value in the above range.
\begin{equation}
    \sigma2 = \frac{1}{SNR}
\end{equation}
The power of each users signal is then transmitted with the following:
\begin{align}
    signal_{power} = \frac{\sum\limits_{i=1}^{X_{k}}|X_{i}|^2 }{X_{K}} \     
\end{align}
and the white gaussian complex noise is modelled as:
\begin{align}
    noise = \mathcal{CN}(0,1)
\end{align}
Finally we arrive at the output of the channel as :
\begin{align}
    Z = y_{h} + \sqrt{\sigma^2} \cdot \sqrt{signal_{power}} \cdot noise    
\end{align}

Lastly, we have the performance metrix MSE, which is a defined as the MSE between the ground truth values (depicted in \ref{fig:MSE}, as the $Tx$) and the received signal (depicted in \ref{fig:MSE} as $Rx$).
\begin{figure}[H]
    \centering
\begin{equation}
  MSE = \frac{\sum\limits_{u_{n,i}=1}^{u_{n,i}}|Tx - Rx|^2 }{u_{n,i}} \
\end{equation}
    \caption{Mean Square Error between Ground truth and outcome.}
    \label{fig:MSE}
\end{figure}

% Next, looking at Kullback Leibler Divergence of a discrete probability mass function (PMF) as a method evaluating the different scenarios, where $x$ is the PMF of each subcarrier in the system. 
% \begin{figure}[H]
%     \centering
% \begin{equation}
%     D_{KL}(P_{Tx}||Q_{Rx}) = \sum\limits_{x\in X} P_{Tx}(x) \log \left(\frac{P_{Tx}(x)}{Q_{Rx}(x)}\right)
% \end{equation}
%     \caption{Kullback Leibler Divergence , Tx PMF, Rx PMF.}
%     \label{fig:MSE}
% \end{figure}

\subsection{Simulation Results}
The following simulations were conducted in two different modes. All simulations consider random white complex Gaussian noise, $w$. First we take a deterministic channel response, $h=1$, which would show to us the behaviour of the system model in the best case channel scenario. Next, we consider a more realistic scenario when the channel is modelled as a response of size 3, which is modelled as a white complex Gaussian. The comparison of the varying affect of the number of users, $k$ and re-transmissions $l$ is considered and evaluated. 
The numerical analysis will compare the affects of ;
\begin{itemize}
    \item TBMA with increasing number of users, $k$
    \item TBMA with varying number of re-transmissions, $l$ without power constraint.
    \item TBMA with adaptive power based on re-transmissions,  $l$ with power constraint.
\end{itemize}

For all methods analysed numerically, the distribution is uniform, where, the probability of each subcarrier being encoded is $\frac{1}{X}$. When evaluating the increasing number of users, it was decided that the best approach would be to keep a ratio of users to subcarriers, 4:1. This has been decided to keep the experiment constant and not introduce other behaviour which may be a by product of varying subcarrier sizes. 
Looking first at the how the number of users would influence the MSE. It would be a reasonable assumption to believe that as the number of users increases, then the MSE should also decrease, which is an interpretation of the parameter estimation in Section \ref{param_estimation}. The idea is that as more users are transmitting information, then more of the noise is ironed out. Lastly, as the SNR range increases to more favourable values. It is also assumed that higher SNR ranges should yield lower MSE values as the noise has less of an impact on the signal. 

\subsubsection{Channel, h=1}

% users
\textbf{Users}
\begin{figure}[h]
    \centering
    \includegraphics[
      width=12cm,
     height=14cm,
    keepaspectratio,]
    {users_change.png}
    \caption{Varying effect of the SNR vs MSE of different number of users in the network. This simulation has 4 re-transmissions and 4 subcarriers fixed. }
    \label{fig:effect_of_users}
\end{figure}

In Figure \ref{fig:effect_of_users}, it can be seen that as we increase the number of users in the network, that the previous assumption of decreasing MSE would hold still. We can also see from Figure \ref{fig:effect_of_users} that the affect of increasing the users is beginning to saturate, slowing as we increase the number of users. This observation is also a sensible assumption, that as the number of users tends toward infinity, we would see more of a true likeness of the original distribution.

% retransmissions

Looking at the first of the two scenarios for re-transmissions, without power adaption. It would be assumed that the effect of the increasing number of re transmissions would decrease the MSE, due to the averaging effect of re-transmitting the same information reducing the error in the estimate to the real distribution. 

\begin{figure}[h]
    \centering
    \includegraphics[
      width=12cm,
     height=14cm,
    keepaspectratio,]
    {rtx_change.png}
    \caption{Varying effect of the SNR vs MSE of different number of re-transmissions of the same channel. This simulation has 64 users and 16 subcarriers fixed.}    \label{fig:effect_of_retx}
\end{figure}

As per the previous assumptions, we can clearly see in Figure \ref{fig:effect_of_retx} increasing affect of the re-transmissions does indeed reduce the MSE of the system, however, this affect is only really seen at higher SNR ranges and the affect does seem to saturate quite quickly. This is due to the channel being deterministic, the effect of the re-transmission has less importance, as the noise $w$ is added after the summation of the signal at the fusion centre, thus there is nothing to average out from the received signal. This approach, does not satisfy the power constraint on transmission.  

\textbf{Adaptive Power Re-transmissions} \newline
Lastly, looking at how the adapting the power across re transmissions affects the over all MSE. The idea is that such a system of TBMA should be constrained by power, it would be inefficient if every re-transmission and subcarrier index of the same signal was sent at full power, this would be a huge overhead for an information centric approach. Thus the idea was adopted such that each user would transmit their respective signal with power,$\frac{1}{L}$, where L is the total number of re-transmissions of the scheme. It would be expected that this approach would be similar to that of full power re-transmissions, where more re-transmissions result in decreased MSE. 

\begin{figure}[H]
    \centering
    \includegraphics[
      width=12cm,
     height=14cm,
    keepaspectratio,]
    % {event_subcarriers_change.png}
    {adaptiv_det_retx.png}
    \caption{Varying effect of the SNR vs MSE of having an adaptive power of each signal transmitted based on the number of retransmission.  }
    \label{fig:adapt_det_retx}
\end{figure}

Looking at Figure \ref{fig:adapt_det_retx}, we can see a similar downward trend in the MSE, however it would appear that using the adaptive power seems to decrease the performance as we increase the number of re-transmissions. This could be due to the deterministic channel and lack of random channel response to average out the transmitted signal.


\subsubsection{Channel, h=random}
Now, looking at the same scenarios, but using a random Gaussian complex channel. The same preconceived ideas should apply for all three use cases, however due to the more realistic channel, we should see more noisy representations of what the deterministic channel showed.

% users
\begin{figure}[H]
    \centering
    \includegraphics[
      width=12cm,
     height=14cm,
    keepaspectratio,]
    {users_randoms.png}
    \caption{Varying effect of the SNR vs MSE of different number of users in the network. This simulation has a 4:1 ratio of users to subcarriers. }
    \label{fig:effect_of_users_random}
\end{figure}

In Figure \ref{fig:effect_of_users_random}, it can be seen that even with a random complex channel that increasing the number of users still decreases the MSE, of course, the MSE is generally much higher in the random channel, thus we observe the initial MSE of the corresponding plots to be different. We also notice that the MSE seems to saturate around the 15dB mark. [To explain why this saturation is seen].

% retransmissions
\begin{figure}[h]
    \centering
    \includegraphics[
      width=12cm,
     height=14cm,
    keepaspectratio,]
    {retx_random.png}
    \caption{Varying effect of the SNR vs MSE of different number of re-transmissions of the same channel. This simulation has 64 users and  1 subcarriers fixed.}    \label{fig:effect_of_retx_random}
\end{figure}


In Figure \ref{fig:effect_of_retx_random}, much like the non power adaptive approach in \ref{fig:adapt_det_retx}, the MSE is decreased overall with the increase of re-transmissions with saturation as the number increases. Again, we observe this initial sharp decrease in the MSE for low SNR values, then coming to a saturation point around 15-20dB. 

\begin{figure}[H]
    \centering
    \includegraphics[
      width=12cm,
     height=14cm,
    keepaspectratio,]
    % {event_subcarriers_change.png}
    {adaptiv_random_retx.png}
    \caption{Varying effect of the SNR vs MSE of having an adaptive power of each signal transmitted based on the number of retransmission.  }
    \label{fig:adapt_random_rtx}
\end{figure}

Lastly, looking at how the adapting the power across re transmissions affects the over all MSE. Figure \ref{fig:adapt_random_rtx} plot is comparable to Figure \ref{fig:adapt_det_retx}, however unlike Figure \ref{fig:adapt_det_retx}, we do not see this decrease in performance when we increase the number of re-transmissions $L$, conversely we see what was expected in the prior assumptions, that as we increase the number of re-transmissions the MSE decrease and the performance gain is noted. 

\subsubsection{Different Schemes}
Now we will look at the comparison of the different schemes which were presented in the Section \ref{schemas_and_channels}. The following numerical analysis of above proposed schemes have different distributions based on the scheme, where the environment has 64 Users, 16 subcarriers and 4 re-transmissions
\begin{itemize}
    \item Scheme \ref{fig:xor_schema} for the data model found in \ref{eq:event01common}, which splits the available subcarrier in half, half used for ACK and the other for NACK. The analysis at the start of each simulation takes a random number between the 0 and $K$ users which is then used to assign the number of ACKS for the simulation. The choice of encoded subcarriers is uniformly distributed across all available.
    \item Scheme \ref{fig:cqi_ind} for the data model found in \ref{eq:event02independent}. This approach chooses a uniform random subcarrier to assign a value to, which equates to a real world CQI value. 
    \item Scheme \ref{fig:cqi_pmf} for the data model found in \ref{eq:event03PMF}. Similarly to the previous approach, a random value is chosen for the CQI value, however this time the simulation takes a normally distributed sample of 1000 values which mean of the random value, this generated PMF is then transmitted along the channel. 
\end{itemize}

It would be expected that all of the schemes have similar MSE across a given SNR range with slight fluctuations depending on the respective encoding. However, it would be expected that channel schemes that are transmitting non correlated information could perform slightly worse as the resolution of the information increases, thus the error would probably increase too. Conversely, it would be a sound assumption that when the users in a group are transmitting common information, the MSE should be decreased. It would also be expected that underlying similarities are more prominent in the deterministic channel vs the random one. 


Both the deterministic channel and the complex random channel have been compared, below in Figure \ref{fig:comp_mse_det}, we can see the results of the deterministic channel. As expected, the correlated information for  both CQI cases outperforms the ACK/NACK use case, which given the prior assumptions would make sense, as the ACK/NACK is transmitting mutually exclusive information, thus the assumption about correlation would not apply here. On the other hand, it can be seen that both CQI use cases perform similarly.

\begin{figure}[H]
    \centering
    \includegraphics[
      width=12cm,
     height=14cm,
    keepaspectratio,]
    {comparison_det_3schemes.png}
    \caption{MSE Vs SNR - comparison of different schemes, deterministic complex channel}
    \label{fig:comp_mse_det}
\end{figure}

Looking now at the random complex channel analysis, seen in Figure \ref{fig:comp_mse_rand}, it can be seen that still the ACK/NACK scheme is still performing the worst. What is notable here, however is that in Figure \ref{fig:comp_mse_det}, the PMF and the Individual CQI scores perform similar. However, when the channel is random complex, we notice that the CQI PMF outperforms the individual CQI scheme. It can be said that is expected, due to the nature of each user transmitting correlated information in the form of a PMF, which has been seen in section \ref{case3}. 

\begin{figure}[H]
    \centering
    \includegraphics[
      width=12cm,
     height=14cm,
    keepaspectratio,]
    {scheme_comp_16sc_64users_4rtx.png}
    \caption{MSE Vs SNR - comparison of different schemes, random complex channel}
    \label{fig:comp_mse_rand}
\end{figure}


\section{Application to the LTE framework}\label{lte_app}
\subsection{Introduction}
In this section the process of implementing the TBMA approach to the Long term evolution (LTE) framework is investigated and discussed, Looking at the different protocol layers and architecture. The given use-case of using TBMA as a feedback mechanism for multicast transmission as of the time of writing this thesis is not yet present. This section will progress through the different aspects of the LTE framework to investigate how such a feature could be consolidated with the current implementation. In section \cref{lte_feedback_current}, we examine how feedback is currently implemented is investigated with respect to CSI and ACK/NACK information, this section will outline how this can be adopted to serve TBMA for a simple use case such as ACK/NACK.

The LTE framework which is an ongoing development project of the 3GPP (3rd Generation Partnership Project), it was first proposed as an international standard in 2004 by NTT Docomo of Japan. Since its inception in 2004, the LTE framework has been developed internationally and has become a standard framework for mobile radio communication, the LTE works as an intermediary framework that connects mobile users to the internet. The scope of this thesis, looking into TBMA as a feedback mechanism from multicast transmission does not exceed the elements of the UE and the eNB in the LTE context, these can be viewed below in the figure.

\begin{figure}
    \centering
    \includegraphics[
      width=6cm,
     height=7cm,
    keepaspectratio,]
    {enb_ue_arch.jpg}
    \caption{General overview of LTE Architecture.}
    \label{fig:lte_arch}
\end{figure}

This simple overview consists of a base station (evolved node b, or eNB) and then all the connected user equipment (UE). The LTE network is responsible for decoding, encoding and exchanging information between the users through the IP network. 

The LTE framework is broken down into different layers which are all responsible for different aspects of the life cycle of the data, within the scope of this thesis, we are concerned with Layer 1 ( the physical layer) and part of Layer 2 (primarily the MAC). 

\begin{figure}[h]
    \centering
    \includegraphics[
      width=6cm,
     height=7cm,
    keepaspectratio,]
    {lte_layers_arch.jpg}
    \caption{LTE protocol layers overview.}
    \label{lte_protocol_stack}
\end{figure}

As seen in Figure \ref{lte_protocol_stack}, the aspects of the LTE stack we are interested in are:
\begin{itemize}
    \item Physical layer (PHY) - The physical layer is responsible for all things physical, modulation, power control, link adaption and the physical encoding of the symbols. It communicates data with the MAC layer and receives control data from the RRC.[ref]
    \item Medium access control (MAC) - MAC is responsible for mapping data between logical channels and transport channels. The MAC handles all the multiplexing from logical to transport channels.[ref] 
\end{itemize}

Now the basic understanding of how the LTE framework works, with attention put on the notable aspects of the stack as discussed above, to move forward with the TBMA approach, we should understand how unicast vs multicast works (also for their respective current feedback). 

Unicast - As the name suggests, this is a one-to-one transmission , where each transmission is a direct channel from the sender to explicitly one receiver, this approach takes the intended users RNTI and scrambles the payload so that the data is only receivable by the intended user. This approach means that for every transmission across unicast a corresponding radio resource is also needed for that, this has some obvious redundancies, given the repeated usage of a radio resource, regardless of multiple users have the same corresponding information. Unicast in the LTE framework is the most frequented method of transmission. If we take each bidirectional arrow as a radio resource for general downlink data and uplink feedback data in \cref{fig:unicast_vs_mutlicast}, we could clearly see how each additional user given the current unicast feedback approach begins to use huge amounts of resources. For the multicast, if TBMA was adopted, only a single resource could be used for mutlicast data and feedback, severly reducing the overhead on the network for grouped data. 


\begin{figure}[h]
    \centering
    \includegraphics[
      width=9cm,
     height=10cm,
    keepaspectratio,]{uni_cast_multi_cast.png}
    \caption{Unicast vs Mutlicast with TBMA,}
    \label{fig:unicast_vs_mutlicast}
\end{figure}

Multicast - This style of transmission, is a one to many relationship. Users are grouped together, then a data transmission is disseminated amongst all users at once. This style of transmission in LTE network can be achieved through a variety of ways. 
\begin{itemize}
    \item First a unicast transmission can be done to all users in a group individually (inefficient use of radio resource as there are many duplicates of the same data based on the group size).
    \item eMBMS - evolved multimedia broadcast service, this approach blocks a slot of the available sub frames and each user in the group has the relevant information to decode the data payload. This approach can simultaneously reach multiple users, but the scheduling of the downlink frames is very inflexible. 
    \item Single cell point-to-multiple point (SCPTM) - this is an amalgamation of the eMBMS and unicast, this approach best utilises the radio resource, whilst reaching multiple users in one transmission.  This approach allows flexible scheduling of the resources and keeps the radio overhead to a minimum. 
\end{itemize}

Feedback for unicast. Currently as of the 3GPP implementation, feedback for unicast transmission is still on a one-to-one topology (can be seen in \cref{fig:unicast_vs_mutlicast}), examples of feedback information the the base station would receive from this one-to-one approach (this is also based on the same RNTI as the DL unicast), would be ;
\begin{itemize}
    \item HARQ ACK/NACK
    \item CQI
\end{itemize}
 
\subsection{Application of TBMA to LTE}

To understand how the an implementation of TBMA could work, we need to understand data progress from human readable format to being encoded and mapped and transmitted along the channel.
To apply TBMA to feedback mechanisms in the LTE framework, the process of how LTE work encoding and decoding the signal must be understood. In modern LTE systems, OFDM (the encoding process of OFDMA discussed in the SoTA) is used to encode data in parallel to be transmitted along the channel. With the process of OFDM in LTE in mind, to realise TBMA a similar sequential encoding and decoding paradigm could be adopted. The previous section we have discussed how to implement TBMA and estimate its parameters, which is the basis of how then to apply this to the LTE framework. Looking at  \cref{fig:tbma_lte_view}, the basic flow of encoding TBMA packets can be seen. 

\begin{figure}[h]
    \centering
    \includegraphics[
      width=6cm,
     height=7cm,
    keepaspectratio,]
    {block_diag.jpg}
    \caption{TBMA LTE overview.}
    \label{fig:tbma_lte_view}
\end{figure}

Like the current OFDM approach, both IDFT/DFT and add/remove CP are still present. However, due to the nature of this information centric multiple user simultaneous transmission approach, we can view the parameter $\theta$ or $PMF$ signatures being sent in parallel along the same channel by multiple users,  unlike OFDM there is no need to equalise or do channel estimation as this is achieved through the matched filter, discussed in the \cref{sys_mod}.[REF]

Next as per \cref{lte_feedback_current}, we can see how the resource elements are encoded as per the usual standard, in the scope of this thesis, a simple mechanism is to be researched to encode the ACK/NACK for multiple users for TBMA. 




\section{Summary and outlook}\label{summary_out}


\addcontentsline{toc}{section}{References}
\bibliographystyle{plain}
\bibliography{\jobname}
\begin{thebibliography}{9}
\bibitem{3gpp36321}
3GPP TS 36.321 Medium Access Control (MAC) protocol specification, Release 15, 2020
\bibitem{3gpp38321}
3GPP TS 38.321 Medium Access Control (MAC) protocol specification, Release 15, 2020
\bibitem{3gpp25319}
3GPP TS 25.319 Enhanced uplink; Overall description, Release 13, 2016
\bibitem{ETSITS136213}
Evolved Universal Terrestrial Radio Access (E-UTRA), Physical layer procedures 
\bibitem{multiple_access_protocols}
Multiple Access Protocols, Performance and Analyisis, Raphael Rom, Moshe Sidi 
\bibitem{shannon_theory}
A Mathematical Theory of Communication, By C. E. SHANNON
\bibitem{information_centric}
Din, I.U., Asmat, H. and Guizani, M. A review of information centric network-based internet of things: communication architectures, design issues, and research opportunities. Multimed Tools Appl 78, 30241–30256 (2019). 
\bibitem{lte_advaned_mobile}
4G: LTE/LTE-Advanced for Mobile Broadband, Erik Dahlman, Stefan Parkvall, Johan Skold
\bibitem{tbma}
G. Mergen and L. Tong, "Type based Estimation over Multiaccess Channels" in IEEE 
\bibitem{access_tech}
Chen, W.K., "The Electrical Engineering Handbook", P.1005, Section 7.1 Access Technologies, 2004.
\bibitem{mod_impact}
Impact of Modulation Schemes on LTE, Juhi Pruthi, Pooja Agarwal, Nikita Jain.
\bibitem{umts_sesia}
LTE – The UMTS Long Term Evolution: From Theory to Practice Stefania Sesia, Issam Toufik and Matthew Baker.
\bibitem{csi_defs}
R\&STS8980 test system analyzes LTE quality indicators: CQI, PMI and RI
WIRELESS TECHNOLOGIES | Conformance test systems. 
\bibitem{fdma_info}
Chen, W.K., "The Electrical Engineering Handbook", P.1005, Section 7.1.1 FDMA, 2004. 
\bibitem{tdma_info}
Chen, W.K., "The Electrical Engineering Handbook", P.1006, Section 7.1.2 TDMA, 2004.
\bibitem{cdma_info}
Chen, W.K., "The Electrical Engineering Handbook", P.1007, Section 7.1.3 CDMA, 2004.
\bibitem{ofdma_info}
Sassan Ahmadi, LTE-Advanced,2014. Orthogonal Frequency Division Multiple Access. Section 9.5
\bibitem{srsLTE}
srsLTE opensource project, https://www.srslte.com/ 
\bibitem{los_pic}
Line of Sight Infographic, https://unlimitedlteadvanced.com/4g-lte/how-do-i-choose-a-4g-lte-antenna/
\bibitem{interference_pic}
http://hydrogen.physik.uni-wuppertal.de/hyperphysics/hyperphysics/hbase/sound/interf.html
\end{thebibliography}
\end{document}

