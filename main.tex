\documentclass{article}
\usepackage[utf8]{inputenc}
\usepackage{amsmath}
\usepackage{graphicx}
\usepackage{float}
\usepackage{amssymb}
\usepackage{amsthm}
\usepackage[printonlyused]{acronym} 
\usepackage[acronym]{glossaries}
\usepackage{hyperref}
\usepackage[capitalise, noabbrev]{cleveref}
\usepackage{subcaption}
\usepackage{ragged2e}
\makeglossaries
\newcommand{\JD}[1]{\textcolor{red}{#1}}
% \cref of cleveref: strip Eq. from equation references
% https://tex.stackexchange.com/questions/122174/how-to-strip-eq-from-cleverf
%\crefname{equation}{}{} % Completely delete Equation, even as name
\crefformat{equation}{(#2#1#3)}
\crefrangeformat{equation}{(#3#1#4) to~(#5#2#6)}
\crefmultiformat{equation}{(#2#1#3)}%
{ and~(#2#1#3)}{, (#2#1#3)}{ and~(#2#1#3)}

\title{Feedback Channels for Multicast Signals in 3GPP LTE and 5G New Radio Networks}


\renewcommand{\maketitle}{
  \begin{titlepage}
  \centering
    \begin{tabular}{lcr}
   \includegraphics[height=1.6cm]{fraunhofer_hhi.png}
      & \hspace{1.5cm}
             \includegraphics[height=1.6cm]{tu_berlin.png}
      \\
      \hspace{1.5cm}
       \includegraphics[height=1.6cm]{eit_digital.png}
       &
        \hspace{1.5cm}
     \includegraphics[height=1.6cm]{uni_trento.png}
    
    \end{tabular}

\begin{flushleft}
    Technical University Berlin \\
    Faculty IV - Electrical Engineering and Computer Science
\end{flushleft}

    {\LARGE Masters Thesis} \\[18pt]
    
    \LARGE \textbf{Feedback Channels for Multicast Signals in 3GPP LTE and 5G New Radio Networks\\
    ACADEMIC YEAR 2020/2021}\\ [22PT]
    
    %  \small
    \begin{flushleft}
    Author : Jonathan Smyth (415261)\\
    \small
    Supervisor TU Berlin : Prof. Dr-Ing. habil. Sławomir Stańczak\\
    \small
    Supervisor University of Trento : Prof. Dr-Ing Daniele Fontanelli \\
    Adviser : Dennis Wieruch, Johannes Dommel\\
    \end{flushleft}
  \end{titlepage}
}

\begin{document}



\maketitle
\newpage 
\tableofcontents

\listoffigures
\listoftables

\newpage

\section*{Acknowledgements}
I would like to acknowledge different parties through out my 6 semesters studying this double masters degree between the University of Trento, Italy and the Technical University of Berlin, Germany. I would like to thank all the professors who have given me valuable insight into the various disciplines that have allowed me to complete this thesis. 

I would like to give a special thank you to the Fraunhofer HHI, for supporting my work during this thesis and in particular to both Dennis Wieruch and Johannes Dommel. For their advice during this time. I would like to thank EIT Digital for creating such an insightful and wonderful experience. Finally I would like to thank my family for their support when times were tough.

I hereby declare that the thesis submitted is my own unaided work. All direct or indirect sources used are acknowledged as references in the bibliography.
For the comparison of my work with existing sources I agree that it shall be entered in a database where it shall also remain after examination, to enable comparison with future theses submitted. Further rights of reproduction and usage, however, are not granted here.

Berlin, Tuesday 7th July 2021.\\
Signed,
\begin{figure}[H]
    \includegraphics[
      width=3cm,
     height=5cm,
    keepaspectratio,]
    {signature.png}
    \label{fig:my_sig}
\end{figure}

\newpage


\section*{Abbreviations}


\begin{acronym}[SPACEEEEEE]
    \acro{LTE}{Long term evolution}
    \acro{CQI}{channel quality indicator}
    \acro{i.i.d}{independent and identically distributed}
    \acro{eNB}{Evolved Node B}
    \acro{UE}{User Equipment}
    \acro{IC}{Information centric}
    \acro{FEC}{Forward error correction}
    \acro{HARQ}{Hybrid Automatic repeat request}
    \acro{ACK/NACK}{Acknowledgement or Not Acknowledgements}
    \acro{PMI}{Precoding matric indicator}
    \acro{RI}{Rand indicator}
    \acro{CSI}{Channel State information}
    \acro{RNTI}{Radio network identifier}
    \acro{PUSCH}{Physical uplink shared channel}
    \acro{PUCCH}{Physical uplink control channel}
    \acro{PDSCH}{Physical Downlink Shared Channel}
    \acro{SIB}{System Information Block}
    \acro{TDMA}{Time division multiple access}
    \acro{FDMA}{Frequency division multiple access}
    \acro{OFDMA}{Orthogonal frequency division multiple access}
    \acro{CDMA}{Code division multiple access}
    \acro{EPC}{Evolved packet core}
    \acro{PMF}{Probability mass function}
    \acro{PDF}{Probability density function}
    \acro{SNR}{Signal to noise ration}
    \acro{UCI}{Uplink control information}
    \acro{DCI}{Downlink control information}
    \acro{UL-SCH}{Uplink shared channel}
    \acro{LSB}{Least significant bit}
    \acro{RRC}{Radio Resource control}
    \acro{CP}{Cyclic Prefix}
    \acro{CSI}{Channel State information}
    \acro{USRP}{Universal software Radio Peripheral}
    \acro{POC}{Proof of concept}
    \acro{OTA}{Over the air}
    \acro{MBMS}{Multimeadia Broadcast Multicast Service}
    \acro{SC-PTM}{Single-Cell Point to Multipoint}
    \acro{MAC}{multiple access channel}
    \acro{MCS}{modulation and coding scheme}
    \acro{MLE}{maximum likelihood estimation}
    \acro{MSE}{mean squared error}
    \acro{ePMF}{empirical probability mass function}
    \acro{SNR}{signal-to-noise ratio}
    \acro{TBMA}{type-based multiple access}
    \acro{crnti}{Cell-Radio Network Temporary Identifier}
    \acro{SDR}{Software Defined Radio}
    \acro{FFT}{Fast Fourier Transform}
    \acro{IFFT}{Inverse Fast Fourier Transform}
    \acro{QoI}{Quantity of Interest}
\end{acronym}
\newpage

\section*{List of Symbols}

x - a scalar of dimension $x$, \\
$\hat{x}$ - Estimate of value.\\
$\tilde{x}$ - empirical measure of value \\
$\boldsymbol{x}$ - vector of size n x 1.\\ 
$\boldsymbol{X}$ - matrix of size m x n.\\ 
$||x||$ - 2 norm of value. \\
$\{1 \dots X\}$ - elements of sets are listed in curled brackets.  \\
$\in$ - member of \\



\newpage

\section*{Abstract}
% wireless comms one line problem descrption. feedback are of utmost opportunity. most chans are limited due to orthogonal principal 
Current LTE systems are limited in the number of UEs they can serve based on the resources available. In LTE feedback is of utmost importance to allow for efficient usage of the limited resources. This thesis investigates a multiple access method for wireless communication systems known as \ac{TBMA} \cite{tbma}, which has been previously proposed as a suitable access method for systems which wish to encode the source and channel jointly. This style of access pays particular attention to groups of users who share a common goal, based on a predetermined scheme. This thesis investigates \ac{TBMA} in the context of feedback mechanisms in \ac{LTE}. Looking at current state of the art for feedback mechanisms in the LTE framework, paying attention to how this can be adopted to serve a \ac{TBMA} style transmission in the context of multicast feedback transmission. We then move a \ac{POC} link layer implementation in a \ac{SDR} environment, using the open-source project srsLTE. Finally we show how TBMA could be a beneficial approach for feedback from multicast groups given the constraint on the number of users a unicast feedback transmission is subject to per \ac{eNB}.  
\newpage

\section{Introduction}\label{intro}
\subsection{Motivation and goal of thesis}
Communication systems like \ac{LTE} and 5G-NR are device centric, which means that they take no consideration into the actual meaning behind the data, ignoring any potential patterns or commonalities. Their primary concern is to acting like pipes for getting data from A to B.
However, as the world is becoming more connected, such areas as Industry 4.0, immersive video/audio and Internet of things begin to prompt more questions about the issues surrounding massive networks and their access methods. As these content-agnostic methods offer the possibility to retrieve granular user specific payloads, which requires resource scaling linearly to the number of users in the network \cite{aloha,graphbased_analysis,coded_ran,capacity_gauss}. In contrast to the aforementioned approach, new communication paradigms are emerging, one such approach would be information-centric. Information centric  medium access protocols are designed to recover information of interest rather than the specific payloads \cite{kountouris,sem_coms,sem_filter,source_chane_coding,source_chane_coding2}. \ac{IC} communication moves away from traditional methods by making the main objective of the protocol a shared goal, rather than just a means of transmitting data. The goal-oriented approach utilises the shared objective of the users signal reconstruction at a fusion centre. 

If we consider applications like inner city autonomous driving in an LTE context, these users are highly dependent on a reliable and efficient communication link between infrastructure and other connected users. \ac{QoI} for autonomous vehicles could be acceleration or velocities. However, in reality in a busy city, the base station would actual want to aggregate all this information from multiple groups of autonomous vehicles to best understand what the real world environment is currently like. Base stations or \ac{eNB}, could aggregate information such as velocities and potentially understand that these grouped vehicles could be e.g in a convoy or are sharing the same direction . Allowing the \ac{eNB} to deduce information from the underlying distribution of the relayed velocities. However, such feedback information or \ac{QoI} in the current \ac{LTE} setup would require a one-to-one mapping of resources which would be inefficient. In order to enable efficient feedback mechanisms of a particular \ac{QoI} for multicast communications, information centric communication approaches, e.g. type based multiple access (TBMA)\cite{tbma}, promises an efficient implementation. This is further justified if a large number of devices with strict latency requirements are taken into account, rendering dedicated feedback channels ineffective. 

An information-centric approach known as \ac{TBMA} is proposed in the context of LTE. We numerically analyse \ac{TBMA} for different numbers of users and re-transmissions, which is to be applied to the current state of the art for LTE feedback methods. Next using an LTE compliant SDR project \cite{srslte} a \ac{POC} implementation of the above investigation about both \ac{TBMA} and Feedback mechanisms will be shown and discussed. 


\subsection{Problem and Context}\label{prob_context}
%  extension of the bullets points etc 
%  can use bullet points here , really just point out which problems i am approaching.
In classical device centric communication, the source and the channel are encoded separately, thus the channel is independent of the data \cite{shannon_theory}. Many different approaches exist for the classical device centric communication approach, \ac{CDMA}, \ac{TDMA} and \ac{FDMA}
\cite{multiple_access_protocols}. However information centric communication (ICC) approaches the problem differently, by jointly encoding both the the source and the channel. ICC utilises the knowledge of both, source  and channel coding, which means that prior to transmitting the information, all parties involved in the ICC transmission are aware of how the channel is encoded and what the encoding scheme means~\cite{information_centric}. The following list outlines some of the major problems at the time of writing this thesis:
\begin{itemize}
  \item In current LTE systems, uplink transmission for such aspects as feedback are achieved through such channels as \ac{PUSCH} or \ac{PUCCH} which are mapped in a one-to-one fashion. Meaning each feedback message consumes one resource, and scales linearly with each user, this is inefficient due to the increasing number of connected devices. 
  \item As more users/devices become connected, more problems arise about how to access the medium efficiently. Conventional access methods begin to introduce bottlenecks.
  \item Current methodologies do not consider potential relationships in the data being transmitted. Situations arise in the real world where, for example multiple users in a group would like to transmit an \ac{ACK/NACK} bit in response to a group message, currently each user would require a dedicated resource and its associated resource costs to achieve this.
  \item Feedback mechanisms such as CQI and ACK/NACK which are crucial for the link adaption to ensure most efficient use of available resources. The LTE framework currently offers feedback options as discussed in \cref{sota}.
\end{itemize}

\subsection{Example: Multicast-Feedback}\label{example_multicast_feedback}

In the following subsection a brief example application is presented to illustrate the approach of \ac{TBMA} in the context of multicast feedback. Multicast is disseminating data to more than one user. Responding to a multicast transmission can be a problem without a suitable access method. As discussed, traditional methods are not without drawbacks regarding resource consumption as groups grow. Conventional access methods aim to mitigate inter-user interference by mandating how to access the radio channel. Without access methods, undesirable destructive interference could result in the final waveform received at the eNB being unusable due to the combining waves destroying each other. Conversely, the interference can also play in the favour of the feedback, waves in phase would have an additive nature on the final reading at the eNB. The introduction of \ac{TBMA} as a jointly encoded source and channel paradigm offers an approach where mitigating inter-user interference is no longer the primary aim. Users know when they encode a particular value, to a particular position, that everyone using this method adheres to the same approach. 

Let us consider the scenario which is depicted in \cref{fig:snr_cqi_plot}, where we have two different groups of users who are subject to receiving multicast data in their respective groups. Visually we can clearly see that there are two distinct groups, one of which is close to the base station and the other group somewhere further away. Now in reality both of these user groups would need to relay some information back to the base station regarding their link quality. Such a metric being the quantized path loss. However if we consider again \cref{fig:snr_cqi_plot}, we can clearly that there are indeed some correlations in the grouped users. In many application, the base station is actually interested in the aggregated information, in this case the spatial distribution of the \ac{QoI} values opposed to the individual.  
\begin{figure}[H]
    \centering
    \includegraphics[
      width=9cm,
     height=9cm,
    keepaspectratio,]
    {Figure_1.png}
    \caption{ Two groups of UEs, transmitting Quantized path loss + shadow fading  }
    \label{fig:snr_cqi_plot}
\end{figure}
In the case of the quantized path loss values, we can exploit this common relationship between the observed data, which results in the underlying physical process. For the given example in this section it is the channel gain that incorporates the path loss, which we can see in \cref{fig:pmfs}. Clearly given what could be the received empirical measure of the data depicted in \cref{fig:pmfs}. We can distinguish between two users groups and based on their transmitted path loss values. We could clearly see that likelihood is that they are close to each other and based on their quantizied path loss, thus recommend a suitable efficient \ac{CQI} value. In an ideal world we want to find the true empirical of the observations across all the users in a group. 
\begin{figure}[H]
    \centering
    \includegraphics[
      width=9cm,
     height=9cm,
    keepaspectratio,]
    {Figure_2.png}
    \caption{Frequency of occurrence of Quantized path in dBm }
    \label{fig:pmfs}
\end{figure}
Consider two similar scenarios in a multicast transmission from a base station, where the base station transmits some information to all its grouped users. First the eNB wants feedback information about for example, the quality of the downlink channel. As TBMA is information centric, each encoded subcarrier on transmission has a particular non arbitrary meaning, which in the case of device-centric may not be true. As we jointly encode the source and the channel, potential interference should be mitigated which could be introduced in device-centric approaches. Second, the base station instead of feedback information, wants normal uplink traffic, which could be any kind of arbitrary data from a user, in this case, more conventional device-centric approaches may be of better use. In this thesis we show that significant gains can be realised in estimation quality and in system resource consumption, if the physical layer and the multiple access are designed jointly for the purpose of estimation, not just explicit value retrieval.

Using information centric approaches such as TBMA, would allow simultaneous usage of the resources available and substantially reduce the latency of the feedback. Users in a massive sensor network would allow for different types of information to be fed back to the base station. In LTE different modulation schemes exist which are used depending on the quality of the wireless channel. Modulations scheme play a key role in adapting to noisy environments and allow for consistent and efficient transmission of data. One such indicator which plays a role in the type of modulation used in data transmission is the \ac{CQI}. Consider the scenario when multiple users receiving unicast transmissions, each UE, could be anywhere in the cell region with unknown obstacles or interference, this means that there exists the situation when different modulation schemes are being used for different UEs attached to the base station. In the case of multicast user groups, it is not possible to modulate a single multi cast in different schemes. Therefore a single scheme is needed for a single multicast to many users, this scenario in itself elicits the same logically inefficient error we see in the aforementioned paragraph as all users in the group would need to relay their own \ac{CQI}. Instead, a better approach to understand the collective consensus of the \ac{CQI} and thus deriving the bare minimum modulation needed for a sufficient transmission would be via the TBMA approach. Each user would transmit their \ac{CQI} value, allowing the \ac{eNB} to adapt their multicast transmission to best suit the users. 

Given the above example, we can see how it is no longer important to understand what every individuals exact measurement is. Rather it is more important to get an overall understanding of how good the channel quality is distributed for the the users in a group. Under this premise, we do not need to know that i.e exactly 15 UEs had a CQI of 6 and 8 UEs had a CQI of 12. In this situation the granularity of the data is less important. Knowing this information can allow sensible choices of modulation schemes by the base stations and better use of radio resources as an overall. At the same time satisfying the overall needs of a group instead of the individualistic.  Overall, the question of use-cases for TBMA is still open, as this is relatively new concept. There exists many areas of research in which \ac{TBMA} could be applied. Recently there has been a big move to incorporate information centric communication into wireless communication, with the 6G project also gaining a lot funding to investigate such methods as TBMA in an information centric communication approach. 

\newpage
\subsection{Thesis Structure }
The thesis is structured as follows; firstly a short introduction will be given in \cref{intro}, next the problem we are facing and context of why this approach is ideal in \cref{prob_context}. With some background motivation understood, we would move onto the current state of the art approaches for feedback in LTE and multiple access methodologies which is found in \cref{sota}, this section presents also the current process in the LTE framework of how feedback mechanism work. Next, we would move then into the \cref{tbma_sec} where the system model for TBMA is presented and discussed. This will give the basis of the mathematical model which is to be understood for TBMA. Data specific models are then presented in \cref{data_models} based on potential use-cases for TBMA, this will illustrate how TBMA could be applied to different data models in the real world. Once the model is understood, the process of how we can estimate the parameters based on our model is then presented in \cref{param_estimation} and discussed, this is broken down based on a random complex channel and an identical channel \cite{tbma}. After we set our theoretic framework, some examples of real world applications of TBMA are presented in an illustrative and informative manner, showing how the different subcarriers and channel structure should look, see \cref{schemas_and_channels}

Next, we would move on to the simulations and numerical analysis of the results of TBMA, comparing different features of the channel architecture that we can exploit to tune our results and optimise the channel, see \cref{sim_results}. Next a discussion on the application to the LTE network in depth and how thee feedback could be achieved given the current LTE implementation, this is discussed and presented \cref{lte_app}. A proof of concept implementation based on the open source project srsLTE is presented and tested in a more realistic environment than the simulation in \cref{srslte_poc}. Finally, future steps and outlook of TBMA are presented in \cref{summary_out}. 

\newpage
\section{Technical Background} \label{sota}
Continuing on from the introduction about the motivation and problems to be investigated in this thesis, this section gives an overview of the conventional orthogonal access methods as used in  LTE/5G. Pointing out how they coincide with the problems which are to be addressed and investigated. We then moving onto one of the main themes of the thesis, \ac{TBMA}. Which is presented in its generic context as a multiple access method, discussing the system model and some notable information we can derive from the equation. The section then moves onto an investigation into which feedback mechanisms are available in the LTE network, how they work from an architectural point of view, what each of the metrics are used for and how the encoding process should look like for periodic and aperiodic transmission.

\subsection{Multiple Access}
In general within wireless communication systems, multiple devices need to share the respective "wireless resources". In order to coordinate this situation, we need to have an access scheme which describes how the resources are shared and then also an access protocol which defines how the devices access the resources, i.e scheduled,random or contention-based etc. A radio resource can be thought of as a proportion of the wireless radio medium which is available, this can be multiplexed by time, frequency or code \cite{access_tech}. Although this thesis is not interested explicitly in traditional multiple access methodologies, some background on how they function is a good basis to understand and highlight the utility of TBMA in the thesis's context. The comparative overview of traditional access and what could be referred to "new age" access will highlight why TBMA is sensible for the context of this thesis. 

\subsubsection{Orthogonal Multiple Access}
Currently there exists three primary methods of orthogonal multiple access (there are many variations on these). Time-division, Frequency-division, Code-division. The following points outline the basic methodology behind these access technologies, paying attention to how they access the medium. \ac{TDMA},  relies on signals which have been digitised and is based on time division multiplexing \cite{tdma_info}. TDMA in essence allocates different time slots for the users of a channel to transmit their data. In TDMA systems, the spectrum of the available radio is divided into user specific time slots, only the allocated user can transmit on that respective slot \cite{tdma_info}.
CDMA - Code division multiple access uses spread spectrum techniques, meaning each user is not allocated a particular frequency, instead they utilise the full available spectrum. CDMA transmissions are encoded with pseudo-random digital sequences, which are separated by orthogonal codes\cite{cdma_info}. The receiver knows the user specific code sequence and uses said sequence to decode the received signal, this is possible because the cross correlation between different users is small. This can be achieved as the bandwidth of the code signal is chosen to be vastly larger than the information signal, the encoding spreads the signal across the entire spectrum \cite{cdma_info}.
FDMA - Frequency division multiple access (FDMA) assigns users an individual frequency band of the entire available wireless channel frequency. Each users' channel is allocated on demand when entering the cell region \cite[Section...]{fdma_info}. All active adjacent frequencies in a cell are divided by a guard band to reduce cross talk between channels. FDMA is based on frequency division multiplexing and each user has a separate frequency for both uplink and downlink \cite{fdma_info}. OFDMA - Orthogonal frequency division multiple access (OFDMA) is the multi user variant of orthogonal frequency division multiplexing (OFDM), where multiple access is achieved by assigning different sub carriers to different users. This approach allows simultaneous data transmission from several users. The OFDMA approach means the radio resource is two-dimensional (2D), contiguous or non contiguous sub carriers span the frequency and at the same time an integer number of OFDM symbols span the time domain\cite{ofdma_info}.

Clearly from the definitions of the above orthogonal access methods, we can see that the primary aim is to allow access to the wireless channel to mitigate potential inference between users. These approaches all assume that the any kind of combination of the individual signals \ac{OTA}, would result in loss of information or corruption. However, as we have seen from our previous example in \cref{example_multicast_feedback}, this may not actually be the case when all the users know the precoding of the channel and source. The aforementioned methods all assume that content is agnostic and has no correlation or common goal in mind. 

\subsubsection{Type Based Multiple Access}

Although \ac{TBMA} is also an orthogonal access scheme, it differs as it takes a goal-orientated approach actually how the devices access the resources. In \ac{TBMA} we still have multiplexing of the signal which separates the available bandwidth into orthogonal fractions. However, we have a non exclusive reuse of these resources which is defined through the TBMA protocol. 

We will present and discuss the system model for TBMA. To begin, we should first look from a high level what exactly is happening. In device-centric systems, the channel is designed to handle data transfer regardless of the source it has come from. Likewise, the source is designed irrespective of how the channel looks. Device-centric approaches may be beneficial for arbitrary data when the information of each specific device is of interest. However, in the case of TBMA, we may not actually be interested in the specific information each user has, rather combined information that can yield metric of interest, such as a \ac{PMF} of values. 

In general terms in TBMA, if we consider that a group of sensors $K$, it is assumed they observe conditionally \ac{i.i.d} data $X_1 \dots X_K $ (see \cref{fig:tbma_general}) given a parameter $\theta$. For conveniences sake, it is assumed that each $X_k \in \{1 \dots R\}$ takes on a discrete value (where $R$ is the total number of potential outcomes in the specific use case), with a \ac{PMF}~${p_{\theta} = [p_{\theta}(1), \ldots, p_{\theta}(R)]}$. The \ac{PMF} belongs to a family $\{p_{\theta}: \theta \in \Theta\}$ where $\Theta \subset \mathbb{R}$ is the parameter space. Each sensor node in the group then transmits a waveform $\boldsymbol{s}_{k,X_k}$, which is constructed based on the sensor nodes index and the observation of that sensor node, $X_k$. To access the wireless channel, each sensor transmits a signature waveform $\boldsymbol{s} \in \mathbb{C}^{N \times 1}$ selected from the signature matrix $\boldsymbol{S} \in \mathbb{C}^{N \times R}$.

There exists an energy constraint that must be satisfied such that \textbf{ $||s_{k}||^2 \leq E$}, which means that the transmission of each node must not exceed the limit of the total energy available for the transmission. The transmitted waveforms are received through a Gaussian multi access channel. The fusion centre then produces and estimate of the $\hat{\theta}$. The over all objective thus of the TBMA is to design the channel waveforms and estimator such that the error of the empirical measure and received signal is minimised (a suitable method could be using the mean squared error)\cite{tbma}. A graphical representation of the above notation and design has been shown in \cref{fig:tbma_general}. Where $h_{1},h_{2} \dots h_{K}$ are the modelled response of the wireless channel which are assumed to be constant for each transmission $k$. Finally, $\boldsymbol{w}$ is the channel noise, which is assumed to be white complex Gaussian noise $\mathcal{CN}(0, \sigma^2)$, where $\boldsymbol{w} \in \mathbb{C}^{N \times 1}$.

% $\boldsymbol{w} \in \mathbb{\mathcal{CN}^{N \times 1}$ 
% WHERE THE ELEMENTS OF W ARE THE CHANNEL NOISE VECTOR, WHICH ARE COMPLEX NORMAL O SIGMA 

\begin{figure}[H]
    \centering
    \includegraphics[
      width=10cm,
     height=10cm,
    keepaspectratio,]{generic_sys_mod.jpeg}
    \caption{TBMA Estimation over Multi Access Channel}
    \label{fig:tbma_general}
\end{figure}

Now we should present and discuss the system model for TBMA. In waveform design, a crucial observation is that the estimator doesn't actually need to know the exact values of $X_1 \dots X_k $, to achieve its optimial performance. In reality, if each of these nodes could deliver a sufficient statistic with their respective transmissions, then there is actually no loss of information. An example of such a statistic would be the empirical measure \cite{tbma} , i.e the type:
% define emperical measure . define it in words...
\begin{align}
    \boldsymbol{\tilde{p}} = \frac{1}{K} (N_1 ,\dots, N_R),
    \label{empirical_measure}
\end{align}

where $N_j =\sum_{k=1}^K \mathbb{1}(X_k = j)$ is the number of sensor nodes in the network that observe $j^4$. $\mathbb{1}$ is the indicator function which indicates that a value takes 1 if it happens and 0 if not. If we let $u_1,\dots,u_R$ be $R$ orthonormal waveforms, and then setting;

\begin{align}
    s_{k,X_k} = \sqrt{E}u_{X_k}, 
\end{align}

which corresponds to letting every node observing $j$ transmit $u_j$ with energy $E$.  We then arrive at the signal at the fusion centre which is presented as:
%  watch out for inconsistency in the terms in the equatsions 

\begin{align}
    \boldsymbol{y} = \sum_{k=1}^K h_k  \sqrt{E}\boldsymbol{u}_{X_k} + \boldsymbol{w}, \label{eq:system_model}
\end{align}

This becomes easier to understand when we consider an identical channel, where all $h_k$ terms are considered as equal to 1. Which then would be simplified to;

\begin{align}
    \boldsymbol{y} = \sum_{j=1}^R \sqrt{E}N_j u_{j} + \boldsymbol{w}, \label{eq:simplified_system_model}
\end{align}

We see that the received signal contains a noisy version of the empirical measure once we have applied a matched filter of $u_1,\dots,u_R$ and scaled by $1\sqrt{E}k$, we arrive at;

%
% after matching filtering here. 
\begin{align}
    \boldsymbol{z} &= K \tilde{\boldsymbol{p}} + \boldsymbol{w}, 
    \label{output_identical_channel}
\end{align}

Therefore in the case of this deterministic channel, the estimate can directly obtain as:
%
\begin{align}
    \boldsymbol{\hat{p}} &\approx \boldsymbol{z} / K.
    \label{simple_estimation_indentical}
\end{align}

In this case, we want to estimate the distribution of the empirical measure. If we consider that we have $1 \dots R$ events, where each waveform transmitted corresponds to one of the available events. Then we arrive at a point where the estimation of empirical \ac{PMF} $\hat{p}$ would be deduced by taking the entire received signal and dividing this by the total number of users, $K$. If we refer to \eqref{empirical_measure}, we can see how the propagation of the identical channel arrives to \eqref{output_identical_channel}, then dividing by the total users $K$ in \eqref{simple_estimation_indentical} coincides to a noisy version true \ac{PMF}.

When we consider the random nature of both the channel and the additive noise (of which we can assume no prior knowledge), one approach to facilitate better results could be re-sampling. Re-sampling could work by performing multiple re-transmissions $l$ of the same observation. This would in effect allow us to reduce the random nature of the channel and noise, by averaging out the total received signal by the total number of re-samples $L$. This approach would improve estimation, however a trade of may be the additional re-sampling and time taken to do so. With the additional re-sampling we would arrive to a system model as (for a fading channel, where the observation is fixed between re-transmission $l$):

\begin{align}
    \boldsymbol{z}^l = \sum_{k=1}^K h_k^l  \sqrt{E}u_{X_k} + \boldsymbol{w}^l, \label{eq:system_model_retrans}
\end{align}
Conversely we would arrive at a different estimation for that of the rayleigh fading channel, by utilising the \ac{MLE}, shown in \cref{glossary}.
\begin{align}
    \boldsymbol{\hat{p}}[j] &= \sqrt{\frac{1}{2L} \sum_{i = 1}^{L}\boldsymbol{z}^i[j]}.\label{eq:empf_h_rand}
\end{align}
\cref{eq:empf_h_rand} stems from the fact that $|z^l[r]| \approx$ Rayleigh($\sqrt{\frac{N_{r}^l + \sigma^2}{2}}$). Again as we had observations for the identical channel, we can draw some observation for the fading channel.

\subsection{LTE feedback methods}\label{lte_feedback_current}

This section details how the feedback metrics such as \ac{CQI} and ACK/NACK work in the current LTE framework. We will investigate how these feedback methods currently work in LTE, giving the background of how they could be adjusted to suit the needs of multicast. In the current LTE framework, both uplink and downlink transmission are possible. Multicast transmission is downlink traffic data which is disseminated from the eNB to multiple UEs, feedback from the multicast transmission is uplink traffic. However, as discussed in the previous sections, we can see there is no dedicated feedback mechanism for multicast transmission. In LTE the eNB is interested in a plethora of metrics which are beneficial for increasing the efficiency of servicing the UEs. As discussed in \cref{prob_context}, multicast transmission introduces a whole set of new problems for managing link adaptation (amongst other things) between large user groups. Current methodologies are not sufficiently suitable to deal with the large overhead needed to serve feedback in a unicast style.

LTE has many different coding schemes which are used based on the quality of the channel. Channel quality is influenced by interference from other cells, noise,  physical obstacles and general other electromagnetic waves.  Link adaptation is a crucial aspect of LTE that allows for the most efficient and robust transmission of data \cite{umts_sesia}[Section 10.2]. Apart from link adaption, LTE has "received receipts", in the form of ACK/NACK - these ACK/NACK fields allow the base stations to understand if the data has been delivered correctly. LTE architecture has many layers for different purposes, in the case of this thesis and feedback, we are most interested in the Physical layer (PHY is responsible for physically encoding and decoding) and the MAC layer (responsible for the multiplexing and multiplexing of data). Looking at the specification from 3GPP we can see how the current breakdown of feedback mechanisms look. Currently, there exists both periodic and aperiodic feedback from the UE to the eNB (for both ACK/NACK and CSI), these types of feedback can either be on the PUCCH or the PUSCH, depending on the type of feedback requested and the configuration \cite{ETSITS136213}[Section 7.2]. 

To understand the feedback mechanism in better detail, an overview of the propagation of the LTE specific channel structure is needed. LTE channels differ from that of a wireless channel, LTE channels are channels denoting how the flow of data propagates within an \ac{eNB} or a \ac{UE}. For the scope of this thesis, in the respect of the LTE network, the uplink and downlink are of note to us. Although there are many channels in the LTE network, we are interested in the physical, transport channel and logical channel. The mapping of these channels and their direction in with respect to uplink and downlink can be viewed in the  \cref{fig:lte_chan_prop}. The \cref{fig:lte_chan_prop} depicts bi-directional data transfer, with more high level information originating in the upper layers and propagating through the different channels depending on the exact use case of the data, this directional aspect also applies in the inverse direction. 
%  make differentiation between channel as in wireless channel and channels in the LTE framework
\begin{figure}[H]
    \centering
    \includegraphics[
      width=8cm,
     height=10cm,
    keepaspectratio,]{uplink_scheme.jpg}
    \caption{LTE channel propagation}
    \label{fig:lte_chan_prop}
\end{figure}

To understand the process of information transfer between the eNB and the UEs, we must take into consideration the flow of the channels. When constructing an uplink feedback message, the data begins at the MAC layer, and is sent along the logical channel within the LTE protocol structure. Next the data is then multiplexed to the transport channel which offers information for the MAC and higher layers, via the UL-SCH. Finally the data reaches the physical channel, where the data is physical encoded onto its blocks and transmitted along the wireless medium. This outlook is important to understand where the origins of the feedback information are coming from. In LTE the physical layer itself does not decided on the feedback information, it is merely responsible for encoding and decoding of the raw signal. The MAC layer on the UE is where the multiplexing of data and feedback indicators are decided upon on. The UE would transmit on the uplink direction from the physical layer the metrics of interest. The UE then depending on the metric, would then construct the relative feedback payload based on the status of received data. The two metrics of note, \ac{CSI} or ACK/NACK are amongst others items processed in the MAC layer \cite{3gpp25319}[Section 7]. The scope of this thesis focuses more on the physical aspects of LTE, thus outlining the main channel of importance below;
\begin{itemize}
    \item PUCCH - used for control data, this data is transmitted in the form of \ac{UCI}. This usually contains information related to Scheduling request, HARQ ACK/NACK and CQI. 
    \item PUSCH - this is the main user data channel, in certain conditions it also carriers the UCI and feedback metrics ACK/NACK and CQI.
\end{itemize}
Channel quality indicators (CQI), are as the name would suggest indicators which represent the quality of the channel at a given point in time, these are used to understand which kind of modulation the eNB should use. ACK/NACK, Acknowledgement and Not Acknowledgement , these pieces of information are transmitted back to the eNB to verify that the data received was not corrupted and its integrity was intact, these features will be explained in greater details in the next section.

\subsubsection{Channel State indicators}
As discussed previously, metrics or indicators about the channel are utilised in the LTE framework. The indicators are used to adjust the configurations of the transmission of data to best serve the conditions that are currently being experienced at that moment in time. 

At a high level we can group these feedback items as channel state information (CSI) , which can be broken down into; CQI - channel quality indicator, this metric gives insight into the quality of the channel at any given time. The quality of the channel can be influenced by many factors, i.e the environment, other users etc. The value of CQI can be between 0 and 15 , with 0 meaning out of range\cite{ETSITS136213}[Section 7.2.3]. The CQI is then used to decide which modulation scheme to use and the relevant transport block size, depending on the quality of the transmission \cite{ETSITS136213}[Section 7.2], we can see how the \ac{SNR} to \ac{CQI} match up in \cref{fig:cqi_to_snr}. 


\begin{figure}[H]
    \centering
    \includegraphics[
      width=8cm,
     height=10cm,
    keepaspectratio,]{snr_to_cqi.png}
    \caption{Mapping of SNR to CQI values in LTE.}
    \label{fig:cqi_to_snr}
\end{figure}

Dependant on the configuration of the UE the aforementioned CQI feedback is realised through different architectural approaches. Both, time and frequency resources and their configurations are chosen by the eNB in the serving cell region, these mechanisms can be either periodic or aperiodic \cite[Section 7.2]{ETSITS136213}. In the event of the both a periodic and aperiodic report being transmitted in the same sub frame, the aperiodic report takes priority and thus, would be only transmitted. Depending on the scheduling mode of the UE and whether the CQI to be transmitted is either periodic or aperiodic would dictate which channel is to be used, we can see the possibilities for these in \cref{tab:cqi_aperiod_sel}
\begin{table}[H]
        \centering
     \begin{tabular}{||c c c||} 
     \hline
     Scheduling mode & Periodic CQI & Aperiodic CQI  \\ [0.1ex] 
     \hline\hline
     Frequency non-selective & PUCCH &  \\ 
     \hline
     Frequency selective  & PUCCH & PUSCH\\
     \hline
    \end{tabular}
    \caption{Periodic vs Aperiodic transmission and their respective channel.}
    \label{tab:cqi_aperiod_sel}
\end{table}
The aperiodic reporting of CQI is done on the PUSCH. On the other hand, periodic reporting regardless of the selective or non selective is done on the PUCCH. The exact type of the CQI is configured on the eNB through the RRC signalling. CQI reporting has a few distinct types. 
\begin{itemize}
    \item Wideband feedback - UE reports a CQI value for the whole system bandwidth available \cite[Section 10.2.1.1]{umts_sesia}.
    \item eNB configured Sub-band Feedback - UE transmits the entire Wideband CQI report, additionally a CQI value for each sub-band is also transmitted.. Sub-band CQI reports are encoded deferentially with respect to the wideband \cite[Section 10.2.1.1]{umts_sesia}.
    \item UE selected sub-band feedback - The UE selects a set of preferred sub-bands of a predefined size within the entire bandwidth.  The UE reports one wideband CQI with one CQI report reflecting the average quality of the predefined size. The UE will also report the position of said sub-bands \cite[Section 10.2.1.1]{umts_sesia}.
\end{itemize}

If the respective CQI field is not reserved for other purposes then the appropriate field bit can be toggled. The combinations of the triggering mechanism for this report can be viewed below \cite[Section 7.2.1]{ETSITS136213}: 

\begin{table}[H]
    \centering
     \begin{tabular}{||c c||} 
     \hline
      CSI request field & Description \\ [0.1ex] 
     \hline\hline
     '00' & No aperiodic trigger  \\ 
     \hline
     '01'  & Aperiodic report for serving cell\\
     \hline
     '10' & Aperiodic for 1st set of serving cells\\ 
     \hline
     '11'  & Aperiodic for 2nd set of serving cells\\
     \hline
    \end{tabular}
    \caption{Aperiodic transmission and their respective request fields.}
    \label{tab:cqi_aperiod}
\end{table}
These respective trigger values are decoded by the UE on the PDCCH with an uplink DCI format in the UE specific search space \cite[Section 7.2.1, Table 7.2]{ETSITS136213}.
It is worth noting that there exists a minimal reporting interval for any aperiodic transmission, which is 1 sub frame. Furthermore, when aperiodic feedback occurs with no associated transport block of user data, the UE shall utilise the PUCCH instead of the PUSCH.

As discussed earlier, periodic reporting of the CQI also occurs in LTE. UEs are semi-statically configured by the higher layers for periodic reporting of such metrics, which are present on the PUCCH apposed to the PUSCH \cite[Section 7.2.2]{ETSITS136213}. Like most features in the LTE framework, there exists an abundance of possible configurations. Like Aperiodic reporting, periodic reporting supports both wideband and ue-selected sub-band, however eNB-selected is not supported. Similarly, the type of periodic reporting is configured by the eNB higher signalling layers, RRC \cite[Section 10.2.1.2]{umts_sesia}. Periodic wideband feedback is similar to that of the aperiodic which is sent along the PUSCH, however the ue-selected sub-band is different. In UE-selected sub-band, the total number of sub-bands $N$ is divided into $J$ fractions called bandwidth parts. The value of $J$ is dependant on the system bandwidth, in this case, one CQI value is computed and reported for a single sub-band from each bandwidth part, along with its respective sub-band index.

PMI - Precoding matrix indicator is used to determine how the individual data streams (layers in LTE) are mapped to the antennas on the UE. Carefully selecting this matrix yields the maximum number of data bits that the UE can receive on across all layers. If the UE knows what the allowed precoding matrix are, then the UE can send a PMI report to suggest to the eNB the most suitable matrix to use \cite{csi_defs}.

RI - Rank indicator,  this is the number of layers and number of different data streams transmitted in the downlink. The aim of an optimised RI is to maximise the channel capacity across the entire available bandwidth by taking advantage of the each full channel rank  \cite{csi_defs}. 

\subsubsection{ACK/NACK}

Additionally the LTE framework has another form of feedback that is also transmitted on the PUCCH. The Hybrid Automatic Repeat Request (HARQ) is also utilised. The HARQ consists of two parts, firstly the ARQ in LTE is a mechanism that as the name suggests repeats the request if an acknowledgement (ACK) is not received by the sender after a predefined timeout period.  After this timeout period the receiver discards the bad packets and the sender will re-transmit. The next step is the Forward Error Correction (FEC), this process takes the malformed data packets and stores them in a buffer until the next transmission, the idea is that 2 or more packets received with insufficient information to decode them alone can be combined together in such a way they can produce a signal that can be decoded \cite[5.3.2 HARQ operation]{3gpp36321}.

The list below outlines the process of encoding an ACK or NACK into the available resources, this is the process based on the specifications from 3GPP.
\begin{enumerate}\label{proc_pucch_configs}
    \item The UE would verify the configuration of the PUCCH \cite[Section 5.4]{36211}
    \item Procedure for determining PUCCH assignment 10.1 36.213 The 3GPP Specification outlines the explicit rules and combinations that the PUCCH can take. This section outlines the combinations of non/simultaneous and single/multiple cell transmission over PUCCH and PUSCH. This section also outlines the information which is to be transmitted in based on the choice of format \cite[Section 10.1]{36211}. 
    \item How the PUCCH bits should be encoded according to Table 5.4.1-1 in Section 5.4.1 of 36.211, this table outlines the modulation scheme to be adopted and the number of bits per sub frame according to the chosen format \cite[Table 5.4.1-1]{36211}.
    \item Finally, how to map PUCCH symbols to physical resources\cite[Section 5.4.3]{36211}.  
\end{enumerate} 

The HARQ mechanism in LTE is encoded into the UCI and carried usually on the PUCCH, however dependant on some conditions can also be carried on the PUSCH. The UCI format dictates the type of feedback metric to be transmitted. The chosen format allows different metrics to be multiplexed into the same message, these combinations can be viewed in \cref{tab:pucch_formats}, \cite[Section 17.3.1.2]{umts_sesia}. 

\begin{table}[H]
\centering
\begin{tabular}{||c| c||} 
\hline
PUCCH Format & Uplink Control Information (UCI) \\ [0.1ex] 
\hline\hline
Format 1 & Scheduling request (SR) (unmodulated waveform)\\ 
\hline
Format 1a & 1-bit HARQ ACK/NACK with/without SR\\
\hline
Format 1b & 2-bit HARQ ACK/NACK with/without SR\\
\hline
Format 2 &  CQI (20 coded bits)\\
\hline
Format 2 & CQI and 1 or 2-bit HARQ ACK/NACK ex CP\\
\hline
Format 2a & CQI and 1-bit HARQ ACK/NACK (20 + 1 coded bits)\\
\hline
Format 2b & CQI and 2-bit HARQ ACK/NACK (20 + 2 coded bits)\\
\hline
\end{tabular}
\caption{PUCCH formats and their properties}
\label{tab:pucch_formats}
\end{table}


%% sufficient to use the coding scheeme not further explaination. the drawbacks being the coherent - use pilots , and a lot of redunant resources being used , 12 bits of information being sent. 
The HARQ ACK/NACK procedure occurs when a the UE receives some user data on the PDSCH. The UE will then perform its integrity check on the received data and depending on the current setting will trigger a ACK/NACK response to the received payload. Upon receiving a payload from the eNB, the UE can send its ACK/NACK response on either the PUSCH or PUCCH, if there is no user traffic in the uplink then the UE will transmit its response on the PUCCH, however, if there is user traffic (this is visible when there is a valid uplink grant from the UE), the response is encoded into the PUSCH \cite[Section 11.4]{lte_advaned_mobile}. The procedures flow can be viewed in \cref{fig:ack_nack}. 

\begin{figure}[h]
    \centering
    \includegraphics[
      width=8cm,
     height=10cm,
    keepaspectratio,]{ack_proc.jpg}
    \caption{LTE procedure for ACK/NACK response.}
    \label{fig:ack_nack}
\end{figure}

When the PUCCH is being utilised to transmit the feedback control data, a region of 2 resource blocks is assigned for the UE. The two resource blocks are mirrored and placed at either sides of the bandwidth to provide some integrity and spread diversity in the signal \cite[Section 11.4.1]{lte_advaned_mobile}. However, each UE having 2 resource blocks during on sub frame is much to large and not very good utilisation of resources, instead these resource blocks are shared amongst multiple UEs in a serving cell, up to 6 UEs can be present in one Resource block in LTE.  We can see from \cref{fig:pucchformats} that each slot (half of a sub frame), has the relevant uplink information doubly encoded at either end. If we consider the feedback for a given UE/set of UEs is transmitted we would see for example that same information is present in both cases of the $m=2$ in the \cref{fig:pucchformats}.

\begin{figure}[h]
    \centering
    \includegraphics[
      width=6cm,
     height=8cm,
    keepaspectratio,]{pucch_formats_rbs.png}
    \caption{Resource Assignment PUCCH Format 1/2}
    \label{fig:pucchformats}
\end{figure}


As we see in \cref{tab:pucch_formats} it is possible to multiplex both the CQI and HARQ ACK/NACK data from a UE to the eNB on the PUCCH. Providing the higher layers have not disabled this feature, when this is disabled the UE may only transmit the HARQ. However, in sub frames where the eNB scheduler has allowed for simultaneous transmission of both metrics a multiplex of both aspects needs to be achieved. In this case there are two situations of note, for normal CP and extended CP \cite{umts_sesia}[Section 17.3.3]. 

Looking at the multiplexing of CQI and HARQ with normal cyclic prefix (DCI format 2a/2b). To transmit a 1 or 2 bit HARQ ACK/NACK together with the CQI, the ACK/NACK bits are not scrambled, but only modulated with either BPSK/QPSK, resulting in a single modulation symbol $d_{HARQ}$.  This single modulation symbol $d_{HARQ}$ is then used to modulated the second Reference Signal (RS) symbol in each CQI slot, meaning the ACK/NACK is actually signalled through the RS itself.

\subsubsection{Capacity Example}
To understand the limits of the feedback in LTE, we should examine for a given use-case the maximum of the number of UEs which could be served by 1 \ac{eNB}. The following presented examples are for unicast style feedback on the \ac{PUCCH} for periodic CQI values. In LTE, we can calculate the number of UEs a single \ac{eNB} can support for a periodic CQI feedback message. Consider, the following; Number of UEs ($K_{UE}$) =  Resource blocks ($RB_{CQI}$) used for CQI reporting x UEs Multiplexed per Resource Blocks ($RB_{UE}$) x CQI reporting periodicity ($P$). 
    \begin{align}
        UEs = RB_{CQI} \times RB_{UE} \times Period
     \label{eq:pucch_capacity}
    \end{align}
    
Where the CQI reporting periodicity (in subframes) are determined based on the parameter cqi-pmi-ConfigIndex given in Table 7.2.2-1A \cite{ETSITS136213}. Lets assume we have reserved 5 resource blocks (eNB Parameter) for CQI reporting and 6 UEs can be multiplexed per resource blocks. Also lets assume that CQI periodicity is 40 ms then the total number of RRC Connected UEs that eNB can support. Based on the above parameters, we can see the total number of UEs is;
\begin{align}
    5 * 6 * 40 = 1200 
\end{align}

Obviously, the periodicity of the CQI feedback and number of resources blocks can vary, however the number of UEs multiplexed per resource block is fixed. Increasing the number of resource blocks and reducing the periodicity of the the CQI feedback could increase the total number of UEs served, but this may lead to other downsides such poor and lag stricken link adaption due to the delay in CQI values being conveyed back to the \ac{eNB}.
\newpage
\section{Type Based Multiple Access in LTE}\label{tbma_sec}
Until now we have presented a generic \ac{TBMA} model and looked a feedback mechanisms in LTE. First, we would point out potential limitations of deriving the empirical distribution of the data. Then we  Next, we refine the system models presented in \cref{eq:system_model}, \cref{eq:simplified_system_model} and \cref{eq:system_model_retrans} in context of the LTE framework. We will address two use cases for \ac{TBMA}, looking at both CQI and ACK/NACK feedback and how \ac{TBMA} can be adopted to serve these mechanisms. Then, how the to estimation empirical distribution of the feedback metric through TBMA in LTE. Lastly, sample encoding schemes for LTE are presented and discussed. 

\subsection{Limitations of TBMA in LTE}
\subsubsection{Finite value range}
When estimating the empirical measure thus trying to fit a distribution to the received data. Deriving the underlying distribution (when/if discovered) we need to consider some practical limitations. To illustrate with an example, lets again consider the scenario of multiple users perform \ac{TBMA} for an arbitrary value. For example, a value of interest has a range of $0 \dots 15$. We may discover that in the wild the underlying distribution of this values could end up being modelled as e.g. Poisson, which can take on values up to $\infty$. 

\begin{figure}[H]
    \centering
    \includegraphics[
      width=6cm,
     height=7 cm,
    keepaspectratio,]{poiss_dist.jpg}
    \caption{Example showing how a Poisson distribution could be cut at $R$, for arbitrary values.}
    \label{fig:cut_poiss_cqi}
\end{figure}

Consider \cref{fig:cut_poiss_cqi}, which shows an example distribution of Poisson. We would notice that in our example, the real world data is limited in the range of values it can take. In this case, it cannot take a value higher than 15.  Conversely as discussed, the likes of Poisson can take values up to infinity. Therefore, in the case of bounded values an event can take on, any values we would infer from our derived distribution that would be beyond this limit of 15 would be cut and placed in the upper most bin we have. Which may lead to some undesirable behaviour in the real world. 

On the other hand, there exists maybe use-cases where the the distribution of interest could contain a larger range of values. Consider again the example of the Poisson distribution as it can tend to infinity. We can see that the physical practicality of having a huge number of subcarriers to serve the individual event is neither sensible nor desirable. This trade of between the granularity and number of events is one to consider.

\subsubsection{Unknown Distribution}
In the previous limitation, we assume that we know the distribution of the data. However, there exists many cases where in fact we do not. In this case, we can introduce a metric to try determine the distance $D$ and fit the distribution of the data. Where the distance, $D$ could be defined as;

\begin{align}
    D = \sum_{i=1}^n |\hat{p}[i] - p_{\theta}[i]|
\end{align}

which is the distance between the original distribution and the distribution of the output of the channel. This distance metric could be \ac{MSE}. 

For example, in a fading channel when the number of users is less than infinity, the estimated \ac{PMF} would describe the current sample at that point. However this sample will not have converged to the true \ac{PMF} as the nature of the random channel and noise is too great. However, with re-sampling over multiple transmission intervals we can reduce this random nature as discussed in \cref{sota} and assign different measures of accuracy (e.g bias, variance or prediction error) to the sample estimates.
Moreover, when we combine the previous behaviour regarding increasing users, $K$ and now the number of re-transmissions $L$ also tending toward infinity the estimated \ac{PMF} would converge to the true \ac{PMF}. However, this is not of practical relevance, due to the associated costs in re-transmitting the signal in such high order of magnitudes needed. Overall, a trade-off exists between how accurate we want to be to the true \ac{PMF} and the associated costs transmitting multiple re-sampling of the same data. 


\subsection{TBMA for specific feedback in LTE}

There exist some constraints and assumptions for following system models. As discussed in \cref{sota}, it is assumed that the energy constraint $||\boldsymbol{s}_{k}^2|| \leq 1$, where the signal of each transmission does not exceed more than 1 energy unit, this means that each user should not transmit their signal with any extra power. This will play an important role later on when we discuss the re-transmission of the signal. For example, if there are multiple re-transmissions for a given \ac{UE}, they should have the power of the entire transmission normalised.  It is also assumed that all noise and channel response (when applicable) is modelled as white complex Gaussian.   

\subsubsection{CQI feedback over TBMA}
If we consider the original \cref{fig:tbma_general}, but now with a \ac{CQI} specific LTE adaption found in \cref{fig:tbma_cqi}. In this case, the we want to use the \ac{TBMA} for \ac{UE} to transmit their respective \ac{CQI} values back to the base-station. In the case of \ac{TBMA} in the LTE context, we will refer to what was formerly known as waveforms to subcarriers.

As discussed, \ac{CQI} can take up to 16 values. It would make sense that to serve LTE  with \ac{TBMA} channel, a scheme with 16 available subcarriers, (which correspond to a real world CQI value) would be suitable. It can be seen in \cref{fig:tbma_cqi} that $CQI_{1},CQI_{2} \dots CQI_{K}$ correspond to each \ac{UE} dependant on the observation of their particular \ac{CQI} value, will choose and encode that corresponding subcarrier and transmit. The same assumptions and definitions of the channel, $h$ and noise $w$, that were discussed in \cref{sota} apply to these terms. 

\begin{figure}[H]
    \centering
    \includegraphics[
      width=10cm,
     height=10cm,
    keepaspectratio,]{cqi_sys_model.jpeg}
    \caption{CQI Estimation over Multi Access Channel}
    \label{fig:tbma_cqi}
\end{figure}

We can see now how this generic \ac{TBMA} model can be built up to incorporate a variety of different real world cases. Under this new adaptation for a specific use case we would arrive to a LTE CQI specific system model for realising \ac{TBMA}. Where in \cref{eq:system_model_cqi}, the relevant terms are substituted out to allow for the source and channel coding to suit that of \ac{CQI}. Where we have already incorporated the re-transmission terms, $l$. 
\begin{align}
    \boldsymbol{y} = \sum_{k=1}^K h_k^l \boldsymbol{s}_{CQI_k} + \boldsymbol{w^l}, \label{eq:system_model_cqi}
\end{align}
Consider the system model presented in \cref{sota}, we arrive at the point where the energy is substituted in place of the waveforms. This can be applied in LTE, where each \ac{UE} transmits with maximum 1 Energy, then we would arrive to;
\begin{align}
    s_{k,CQI_k} = \sqrt{E}u_{CQI_k}
\end{align}
If we consider the matrix of available event choices in respect to CQI, we can see how each column in this $\textbf{U}^{CQI}$ matrix corresponds to the CQI values 0-15. The IDFT has been applied here as we are talking in the freuqeuncy domain.
\begin{align}
    \text{IDFT}(\boldsymbol{U}^{CQI}) = 
    \begin{pmatrix}
     1 & 0 & \hdots & 0 & 0\\
    0 & 1 & \hdots & 0 & 0 \\
    \vdots & \vdots & \ddots & \vdots & \vdots \\
    0 & 0 & \hdots & 1 & 0 \\
    0 & 0 & \hdots & 0 & 1 \\
    \end{pmatrix}
    \label{fig:sig_mat_cqi}
\end{align}

\subsubsection{ACK/NACK feedback over TBMA}\label{data_models}

More over, we can apply the same setup to another LTE specific use case, when each user in the group would transmit a ACK or NACK response, as described in \cref{lte_feedback_current}, where again the terms $ACK_{1},ACK_{2} \dots ACK_{K}$ denote the observation at each user taking the possible values ACK or NACK. Again, the channel, $h$ and noise $w$ are as described in the \cref{sota}.

\begin{figure}[H]
    \centering
    \includegraphics[
      width=10cm,
     height=10cm,
    keepaspectratio,]{system_model_ack_tbma.jpg}
    \caption{ACK/NACK Estimation over Multi Access Channel}
    \label{fig:tbma_ack}
\end{figure}
Again, as with the use-case of CQI values, we can have a similar approach for ACK/NACK values, where we change the dimensions of the signature matrix and what is transmitted, where we would arrive;
\begin{align}
    \boldsymbol{y} = \sum_{k=1}^K h_k^l \boldsymbol{s}_{ACK_k} + \boldsymbol{w}^l, \label{eq:system_model_acks}
\end{align}
As before, we would substitute in the energy, where each transmission contain 1 Energy. Thus we would attain
\begin{align}
    s_{k,CQI_k} = \sqrt{E}u_{CQI_k}
\end{align}
Each user, $k$ would again dependent on their observation encode and transmit the corresponding subcarrier with their total energy for that TBMA transmission, thus we would have our signature matrix in the case of ACK/NACK as;
\begin{align}
    \text{IDFT}(\boldsymbol{U}^{ACK}) = 
    \begin{pmatrix}
    1 & 0\\
    0 & 1 \\
    \end{pmatrix}
    \label{fig:sig_mat_ack}
\end{align}

From the above two examples we can see how we can begin to apply TBMA in the context of LTE for different use-cases. 
\newpage
\section{PMF Estimation of LTE-based Feedback using TBMA}
\subsection{Identical Channel}\label{det_chan}
It is assumed that the data is i.i.d between sensors but constant during different observation periods. For the specific case of identical channels ${(h^i_k = 1, \forall k)}$, the output of the bank of matched filters in \cref{eq:mf_output} contains a noisy version of the histogram, scaled by the number of users, which is presented in \cref{simple_estimation_indentical}. However we have the additional term in the LTE specific use-case, $l$, which is the number of re-transmission, thus we have the LTE specific estimator for an identical channel of:
%
\begin{align}
    \boldsymbol{z}^l &= K \tilde{\boldsymbol{p}}^l + \boldsymbol{w}^l, 
\end{align}
%
where $\tilde{\boldsymbol{p}}^l = \frac{1}{K}[N_1^l, \ldots, N_R^l]$ is the \emph{empirical measure} or \emph{type} with
\begin{align}
    N_j^l = \sum_{k=1}^K(X^l_k = j)
\end{align}
and $\boldsymbol{w}^l \sim \mathcal{CN}(0, {\sigma^2)}$. Which is the complex Gaussian noise, where .
%
Therefore in the case of this deterministic channel, the estimate can directly obtain the histogram of either CQI or ACK/NACK values from the $l$-th measurement as:
%
\begin{align}
    \boldsymbol{\hat{p}^l} &\approx \frac{\boldsymbol{z}^l}{K}. \label{eq:empf_h_const}
\end{align}

As discussed in \cref{sota}, the same assumptions apply when the number of users, $k$ and re-transmissions $l$ tend toward and away from infinity.
Although the deterministic channel is not a well rounded picture of the real world, it does however give us some insight to underlying aspects of the systems as described in the previous two a paragraphs. This information can help build up a better picture of TBMA moving forward looking into a more realistic situation in the wild.

\subsection{Fading Channel}\label{rand_chan}
% add in info on fading channels here
In the real world, it is more often than not less likely that a transmitted and receiver are in a direct line of sight (direct line of sight is a transmission path that is unobstructed or reflected by obstacles in the environment). An example of line of sight can be seen in \cref{fig:line_of_sight}.

\begin{figure}[H]
    \centering
    \includegraphics[
      width=6cm,
     height=7 cm,
    keepaspectratio,]{line_of_sight.png}
    \caption{Line of Sight obstruction from obstacles \cite{los_pic}. }
    \label{fig:line_of_sight}
\end{figure}

What tends to happen is that in busy environments there are obstacles such as buildings and walls that will obstruct the path of the disseminating wireless signal. The situation leads to multi path waves arriving at the receiver at different time intervals and power due to the physical affect the aforementioned obstacles have on the physicality of the waves.

This means that slightly varying waveform of the original signal are received, these can be out of phase and can be viewed from an $x$ and $y$ component, having both real and imaginary values, which can be summed to give a complete imaginary component. This effect can be viewed in Figure \ref{fig:rayleigh_fade_comps}, where the dotted lines are the different aspects of the signal received and the final complex component in bold. 

\begin{figure}[H]
    \centering
    \includegraphics[
      width=6cm,
     height=7 cm,
    keepaspectratio,]{rayleigh_graph.jpg}
    \caption{Rayleigh fading components of signal. }
    \label{fig:rayleigh_fade_comps}
\end{figure}


The amount of fade for each path which is received at the user is considered to be i.i.d and based on the central limit theorem, when adding up multiple random variables for each of the $x$ and $y$ components we attain a Gaussian PDF for each of the axis. If we then consider that each of these Gaussian are independent, the overall PDF can be obtained by multiplying both PDFs together. 

With channel coefficients drawn \ac{i.i.d}. between sensors and observation intervals $h^l_k \sim \mathcal{CN}(0, 1)$, the output of the matched filter is a vector of random variables, where the \ac{PDF} of the $j$-th element is given by $p_{\mathbf{z}}(\boldsymbol{z}^l[j]; N_j)$, with  
%
\begin{align}
    p_{\mathbf{z}}(x; \sigma) &= \frac{x}{\sigma^2} e^{\frac{-x^2}{2\sigma^2}} ; 
    \newline 0 < x < \infty
\end{align} 
% lambda is mean of pois
% should i do the proof here for MLE of Rayleigh?
Where the likelihood function of the approach is detailed in \cref{glossary}.
Hence, rearranging, the unbiased ML estimate of the $j$-th element of $\tilde{\boldsymbol{p}}$ can be obtained after observing $L$ transmissions as
%
\begin{align}
    \hat{p}[j] &= \sqrt{\frac{1}{2L} \sum_{l = 1}^{L}\boldsymbol{z}^l[j]}.\label{eq:lte_empf_h_rand}
\end{align}

In other words, with a noisy, random complex channel, we can exploit the noise over multiple re transmissions of the exact same information to allow the estimator to reduce the effect of the stochastic noise allowing a more accurate reading to be obtained. As with the identical channel, we observe the same behaviour when we tend toward and away from infinity, as discussed in \cref{sota}.

\subsection{Conceptional Schemes} \label{schemas_and_channels}
The idea of how best to form the TBMA channel to maximise the likelihood of estimation of the empirical data. Different use cases exist for best utilising the channel. This next section describe potential examples of application in the context of the LTE framework. First, a generic event based encoding is shown. Next examples are shown for, ACK/NACK encoding, \ac{CQI} of individual values and how they look from an estimation point of view.

\begin{figure}[H]
    \centering
    \includegraphics[
      width=10cm,
     height=12cm,
    keepaspectratio,]{general_event.jpg}
    \caption{PMF of all users results of feedback across available subcarriers.}
    \label{fig:generic_events}
\end{figure}


The Schema presented in figure \ref{fig:generic_events}, shows the generic approach for taking any event and encoding it to any given available subcarrier. If we refer back to the system model presented in \cref{eq:system_model}, we can see clearly in \cref{fig:generic_events}, how the terms in the system model line up. It now becomes more apparent how the joint channel and source work in unison. We can clearly see in \cref{fig:generic_events}, that multiple users can choose a one-hot encoding of the available signatures (where one-hot encoding means that each user chooses only one subcarrier to encode), depending on the observation that respective user has made, in simplistic terms we can see the summed outcome of all users choices. Referring to \cref{fig:generic_events}, we can see how this $\sum K$-th element without a channel or noise would translate to a \ac{PMF}. Thus, through all users adhering to the same scheme, we arrive at the bottom of the diagram at the fusion centre where we receive the normalised combination of all the transmitted waveforms from all the users in a group, where the normalised sum of all waveforms received would equate to $1$. The resultant \ac{PMF} which we would estimate would look similar to \cref{fig:pmf_subcarriers} 

\begin{figure}[H]
    \centering
    \includegraphics[
      width=8cm,
     height=12cm,
    keepaspectratio,]{cqipmf_subcarrier.jpg}
    \caption{PMF of total roved signal $\boldsymbol{z}$ .}
    \label{fig:pmf_subcarriers}
\end{figure}

The \cref{fig:generic_events} has the same notation as \cref{eq:system_model} to help visually understand how the equations fit together. The same constraints for the equations apply. Any user in the network can take any available subcarrier, $\textbf{s} \in \textbf{S}$, and encode a single bit which correspond $\{1\dots R \}$. These events could correspond to any real world values, providing all the users in the network assume the same encoding style. During for example,  a multicast it would be beneficial for the \ac{eNB} to understand if the majority of users in a multicast group have received the message correctly. 

As in ACK/NACK feedback in unicast, a simple ACK or NACK response could be elicited by each user $k \in K$ for the eNB to understand if the message was received by all \ac{UE} and whether that information was received correctly, or not. However given current feedback approach described in \cref{sota}, using the \ac{PUCCH} or \ac{PUSCH} for each user in the group, this would entail as discussed in \cref{prob_context} a single resource for each user in the group, which as we have shown is not ideal.
 
Consider, a group of $1000$ \ac{UE}, $K$, where each UE transmit their feedback with current uplink feedback methods discussed in \cref{lte_feedback_current}. Under this approach each UE would take time $t$ to transmit each ACK/NACK response on a dedicated channel. This may not be such a issue for small to medium sized groups. However, for larger groups, this would be a time consuming, resource intensive task. Contrarily, using the likes of TBMA could mitigate these resource issues. The current approach in LTE has a one to one mapping of resource to UEs, using TBMA we could have a many to one mapping - i.e. $1000$ users would take collectively time $t$, apposed to needing time $t$ for each user. In LTE ACK/NACK is an automatic response, mixing this with multicast transmission becoming more common we can clearly see time $t$ vs number of UEs $t$ is a huge advantage as the number of users in a group grows. 

\begin{figure}[H]
    \centering
    \textbf{Schema: Encoded ACK/NACK}\par\medskip
    \includegraphics[
      width=8cm,
     height=8cm,
    keepaspectratio,]{ack_nack_lte.jpg}
    \caption{ACK/NACK encoding on 2 subcarriers}
    \label{fig:acknack}
\end{figure}

\cref{fig:acknack}, shows a simple coding scheme where two available subcarriers could encode exclusively either an ACK or NACK depending on what the specific UE has observed at that point in time.

If we take the example of autonomous vehicles in a smart city, where there are a lot of users, pedestrians, potential interference. In this kind of environment, it may be understandable that some data received is corrupt and incomplete. In these safety critical situations, where inevitably there will be multiple sets of grouped users who would be receiving many messages about updates in their environment. This kind of scenario clearly needs to have ACK/NACK feedback to ensure that dropped message and successfully re-transmitted and received, otherwise incurring potential dangerous situations. 


As discussed in \cref{sota}, a very useful piece of feedback information each UE can give the eNB is the Channel quality indicator (CQI), which was discussed in \cref{lte_feedback_current}, The CQI is, as the name suggests an indication of the quality of the channel each user is experiencing. This CQI value allows the eNB to best understand which modulation scheme to use for transmission. The CQI value ranges from 0-15 (modulation from QPSK to 64QAM), see \cref{sota}. This is a very useful tool for the eNB to understand how efficiently depending on the external environmental factors to encode the information along the channel. 

The same situation applies here as the ACK/NACK scenario, the users should best use the channel to convey the collective consensus on the quality of the channel, with potentially huge amounts of users in a cell region, the feedback from multicast could be hard to handle. Using a TBMA with 16 subcarriers, each user could transmit an encoded bit in their respective subcarriers which corresponds to a real world value for the CQI. \cref{fig:sig_mat_cqi}, is illustrated here, each user has a common set of different values which can be taken. 

\begin{figure}[H]
    \centering
    \textbf{Schema: CQI Individual structured encoding}\par\medskip
    \includegraphics[
      width=12cm,
     height=14cm,
    keepaspectratio,]{cqi_events.jpg}
    \caption{Transmitted encoding of user gathered CQI information. }
    \label{fig:cqi_ind}
\end{figure}

As seen in Figure \ref{fig:cqi_ind}, each available subcarrier corresponds to a real world CQI value. This approach allows the eNB to get a snapshot of the current quality of the channel from multiple users. The eNB would receive a PMF of the group of users containing their CQI values at that instant. This would provide an overview of how all users in a group are receiving in the channel and allow the eNB to more appropriately adjust its encoding scheme to suit the majority of users. Circling back to one of the huge benefits of TBMA, as with traditional feedback methods. This underlying distribution of the values would allow the \ac{eNB} to assess what is a suitable trade of for modulation.

\newpage
\section{Application to the srsLTE framework}\label{lte_app}
The previous sections have discussed and presented how TBMA could be adopted to serve the transmission of both CQI values and ACK/NACK values. However we would need to understand how TBMA would fit into the current LTE framework, in this section we will introduce the different ways a multicast transmission can be achieved, giving an overview of the three possible methods of multicast in the LTE framework. Then the process of implementing TBMA approach in the Long term evolution (LTE) framework is investigated and discussed, looking at the different protocol layers and architecture. The given use-case of using TBMA as a feedback mechanism for multicast transmission as of the time of writing this thesis is not yet present.  We will also outline how to achieve multicast ACK/NACK feedback through format 1a in the PUCCH via \ac{TBMA}.

\subsection{LTE architecture}
The LTE framework which is an ongoing development project of the 3GPP (3rd Generation Partnership Project), it was first proposed as an international standard in 2004 by NTT Docomo of Japan. Since its inception in 2004, the LTE framework has been developed internationally and has become a standard framework for mobile radio communication. The scope of this thesis, looking into TBMA as a feedback mechanism from multicast transmission does not exceed the elements of the UE and the eNB in the LTE context, these can be viewed below in \cref{fig:lte_arch}.

\begin{figure}[H]
    \centering
    \includegraphics[
      width=6cm,
     height=7cm,
    keepaspectratio,]
    {enb_ue_arch.jpg}
    \caption{General overview of LTE Architecture regarding \ac{UE} and \ac{eNB}}
    \label{fig:lte_arch}
\end{figure}

This simple overview consists of a base station (evolved node b, or eNB) and then all the connected \ac{UE}, which could be an entity connected to the LTE network which is not an eNB. The LTE network is responsible for decoding, encoding and exchanging information between the users through the IP network. 

The LTE framework is broken down into different layers which are all responsible for different aspects of the life cycle of the data, within the scope of this thesis, we are concerned with Layer 1 ( the physical layer) and part of Layer 2 (primarily the MAC), these entities correspond to the layers spoken about in \cref{lte_feedback_current}. 

\begin{figure}[H]
    \centering
    \includegraphics[
      width=6cm,
     height=7cm, 
    keepaspectratio,]
    {lte_layers_arch.jpg}
    \caption{LTE protocol layers overview.}
    \label{lte_protocol_stack}
\end{figure}

As seen in \cref{lte_protocol_stack}, the aspects of the LTE stack we are interested in are:
\begin{itemize}
    \item Physical layer (PHY) - The physical layer is responsible for all things physical, modulation, power control, link adaption and the physical encoding of the symbols. It communicates data with the MAC layer and receives control data from the RRC.
    \item Medium access control (MAC) - MAC is responsible for mapping data between logical channels and transport channels. The MAC handles all the multiplexing from logical to transport channels. The MAC layer is responsible for the 'when' the feedback mechanisms is used. 
\end{itemize}

Now the basic understanding of how the LTE framework works, with attention put on the notable aspects of the stack as discussed above, to move forward with the TBMA approach, we should understand how unicast vs multicast (SCPTM and eMBMS) works.

In unicast the base station (eNB) sends distinct messages to each \ac{UE} as depicted in \cref{fig:unicast}. To allow this individualistic message transmission each \ac{UE} must have its own unique C-RNTI (Cell - Radio Network Temporary Identifier) as an identifier. The \ac{crnti} is used to scramble the downlink traffic (through the PDSCH) on a sub frame basis for that specific \ac{UE}. That means, that no other UEs in that cell region can decode that respective sub frame, instead any information which is not specific that UEs particular \ac{crnti} is ignored. The CRNTI is attain by each UE when it enters a cell-region \cite[Table 7.11]{3gpp36321} or in the random access procedure.  

\begin{figure}[H]
    \centering
    \includegraphics[
      width=7cm,
     height=9cm,
    keepaspectratio,]{unicast_fig.jpg}
    \caption{Unicast Dissemination}
    \label{fig:unicast}
\end{figure}

The LTE framework is not limited to unicast transmission, two forms of multicast are also possible. Multicast is a one to many relationship. Users are grouped together, then a data transmission is disseminated amongst all users, this style of transmission is . This style of transmission in LTE network can be achieved through a variety of ways. There both \ac{MBMS} and \ac{SC-PTM}. Firstly we will look at MBMS, which unlike unicast utilises another primary channel called the physical multicast channel (PMCH), which is used for evolved multimedia broadcast multicast service (eMBMS). One notable difference between these two approaches is that eMBMS cannot be multiplexed with unicast, this is due to the structure of their sub frames differing. The multicast data from eMBMS is usually transmitted from many synchronised base station at the same time, this mode is called Multicast Broadcast Single Frequency Network (MBSFN). Finally, in LTE there exist a new style of multicast transmission, which is known as \ac{SC-PTM}, which can be viewed as a combination of \ac{MBMS} and unicast. Where the SC-PTM channel utilises the \ac{PDSCH} apposed to the PMCH as in MBMS. The idea of this approach is such that unicast and multicast data can be multiplexed in the same sub-frame. This has some very notable advantages compared to that of MBMS. SC-PTM uses the same methodology as unicast, however as in unicast each \ac{UE} has a \ac{crnti} which limits transmission to that specific RNTI for each individual UE, SC-PTM utilises a SC-RNTI which is attained through the \ac{SIB}20 for any interested grouped data. This SC-RNTI is then used to un-scramble the relevant sub-frame which contains the information needed to acquire the group specific RNTI known as the G-RNTI. This G-RNTI is then used like any other C-RNTI by a UE, with the notable difference that multiple UEs have it. So in effect SC-PTM is like a unicast in regards to dedicated resources being used for a transmission, but the scope of the G-RNTI is for multiple users. SC-PTM also allows for fully flexible scheduling and is not constraint to specific frames or regions of the PDSCH like MBMS is. This is a more efficient method to disseminate grouped user data vs \ac{MBMS}. As we can see with SC-PTM, LTE has already begun to look at better methods to utilises one dedicated resource for multiple users. This approach ties in well with \ac{TBMA} as a method to reduce the resource radio overhead where possible for grouped transmission. Consider an example where multicast happens through SC-PTM and the feedback from the multicast is accomplished through \ac{TBMA}. In this scenario, we would arrive at true representation of \cref{fig:mutlicast}, where each arrow depicts a fully flexible one-to-many and many-to-one relationship of the downlink and uplink transmission in the LTE context, in the aforementioned situation, we would complete the circle for efficient multicast transmission.  

\begin{figure}[H]
    \centering
    \includegraphics[
      width=7cm,
     height=9cm,
    keepaspectratio,]{multicast_fig.jpg}
    \caption{Multicast Dissemination}
    \label{fig:mutlicast}
\end{figure}


\subsection{Application of TBMA to srsLTE}

To understand how the an implementation of TBMA could work, we need to understand data progress from human readable format to being encoded and mapped and transmitted along the channel.
To apply TBMA to feedback mechanisms in the LTE framework, the process of how LTE work encoding and decoding the signal must be understood. In modern LTE systems, OFDM (the encoding process of OFDMA discussed in the SoTA) is used to encode data in parallel to be transmitted along the channel. With the process of OFDM in LTE in mind, to realise TBMA a similar sequential encoding and decoding paradigm could be adopted. The previous section we have discussed how to implement TBMA and estimate its parameters, which is the basis of how then to apply this to the LTE framework. Looking at  \cref{fig:tbma_lte_view}, the basic flow of encoding TBMA packets can be seen. 

\begin{figure}[h]
    \centering
    \includegraphics[
      width=8cm,
     height=9cm,
    keepaspectratio,]
    {block_diag.jpg}
    \caption{TBMA LTE overview.}
    \label{fig:tbma_lte_view}
\end{figure}

Like the current OFDM approach, both IDFT/DFT and add/remove CP are still present. However, due to the nature of this information centric multiple user simultaneous transmission approach, we can view the parameter $\theta$ or $PMF$ signatures being sent in parallel along the same channel by multiple users, unlike OFDMA there is no need to equalise or do channel estimation as we are more interested in the correlation of the values which have been sent independently from the users and thus inherently been summed \ac{OTA}.

Next as per \cref{lte_feedback_current}, we can see how the resource elements are encoded as per the usual standard, in the scope of this thesis, a simple mechanism is to be researched to encode the ACK/NACK for multiple users for TBMA.

As the scope of this proof of concept implementation in this thesis is limited to Format 1A on the PUCCH, a graphical representation of this process has been shown in \cref{fig:pucch_encode_decode_tbma}. The process for encoding an ACK/NACK on the UE side would remain the same, the UE would encode this 1 bit piece of information into the symbols highlighted in \cref{fig:pucch_encode_decode_tbma}, this PUCCH is then put through an \ac{IFFT} and sent along the wireless channel, on the eNB side the reverse is then done, \ac{FFT} and decode the of the signal. However at this point is where the TBMA would be divergence from the usual approach of the feedback over unicast. At this point after the PUCCH is begun to be decoded, we would take these scalar values of the ACK/NACK which are converted from raw symbol to real values and used as correlations to tell whether the received value is indeed an ACK or NACK, these are then summed and normalised as per the original TBMA equation in \cref{eq:system_model} .


\begin{figure}[H]
    \centering
    \includegraphics[
      width=11cm,
     height=17cm,
    keepaspectratio,]
    {tbma_pucch_prop.jpg}
    \caption{TBMA srsLTE process.}
    \label{fig:pucch_encode_decode_tbma}
\end{figure}

With the architectural aspect of \ac{LTE} discussed in the previous sections we could potentially suggest a few modifications to the its structure to incorporate \ac{TBMA}. Obviously the LTE framework has many modes of operation and can be dynamic in its assignment of resources depending on the configuration from an \ac{eNB}. However the \ac{UE} could on the physical layer be adopted to encode a specific area of the PUCCH, to perform TBMA on the fly dependant on begin part of a group or not. As with the likes of ACK/NACK feedback which is constructed in the \ac{MAC} layer, could be extended to do the same for multicast users, propagating this configuration on down to the PHY layer as it does with normal unicast. The likes of the \ac{eNB}, could be adopted to include a fusion centre to consume the multicast feedback. The likes \ac{SC-PTM}, could be extend to include configurations in its \ac{SIB}20 to allow for the necessary parameters to notify the \ac{UE} to get its self ready for \ac{TBMA} transmissions.



\subsection{Proof of concept}\label{srslte_poc}
Until now we have discussed theoretically how we could incorporate TBMA into the LTE framework with respect to the physical process. We would now present and discuss a walk through implementation. The Proof of concept (POC) was done in an LTE compliant software framework, srsLTE. The srsLTE project is modular library and full stack SDR implementation on the LTE network. This is LTE compliant and adheres to the standards dictated in the 3GPP specifications. The srsLTE project has developed a full stack, including an \ac{EPC}, \ac{eNB} and \ac{UE}. They also have exposed a modular library which can be used to piece together different aspect from within the full stack. They also offer many example code files, which can be adapted to suit the users needs \cite{srsLTE}. 

The people at srsLTE have exposed smaller unit cases for subsystems within each of the stacks. They allow quick prototyping and proof of concept coding, in particular this POC was based on their example file for testing configuration on the PUCCH. Which is an example of how the PUCCH ACK/NACK encoding/decoding is LTE operation. This example does not transmit any data over a physical wireless medium, rather it uses the same functionality as the production version of the stack until the point where the data is transmitted across the actual USRP. This means that there is no physical channel and if left as is, would equate the same scenario as described in \cref{det_chan}. However this can be adapted by programmatically adding in a random complex channel as described in \cref{rand_chan}.

This example file allows for prototyping different configurations which are validated by the srsLTE stack functions. The modular functions used allow the user to try all the combinations of the PUCCH or PUSCH configurations based on the 3GPP specification, 36.211 Section 5.4, as described in the earlier sections. The example file follows the same setup procedure which would be found in the fully functioning UE/eNB setup for encoding and decoding the PUCCH/PUSCH - we are however, only interested in the PUCCH for this scope. The relative aspects of the eNB and UE are initialised with their respective configurations. The setup of resource configurations of the tests were as follows:

\begin{itemize}
    \item Physical resource blocks = 6
    \item Transport blocks = 1
    \item Carriers = 1
\end{itemize}


It is worth mentioning that referring back to \cref{sota}, where we present an example of maximum capacity of UEs transmitting periodic CQI values on the \ac{PUCCH} for 1 \ac{eNB}, which has similar setting this link level simulation was 1200 UEs. 
After this, each UE is to encode just one ACK or NACK message at random, a comparable situation as found in \cref{schemas_and_channels}. This data is is then sent through the pipeline as would be encoded in the real world in an LTE network. The approach initially follows the same procedure for encoding the ACK/NACK bits to the PUCCH on Format 1A, however, some changes needed to be made to accommodate TBMA style feedback.

The steps to be followed are the same as shown in \cref{sota}. TBMA would follow the similar steps as \cref{encode_pucch_proc} with a few caveats to understand adhere to TBMA principals. With TBMA, the available subcarriers for the transmission in the case of the ACK/NACK have predefined meanings, i.e 1st available signature is for ACKs, second available for NACKs.

\begin{figure}[h]
    \centering
    \includegraphics[
      width=10cm,
     height=7cm,
    keepaspectratio,]{acknack_creation_encode.png}
    \caption{Creation of ACK/NACK message to be encoded.}
    \label{fig:encodeACKSNACKS}
\end{figure}

\cref{fig:encodeACKSNACKS}, details how for example in this proof of concept file, 10 UEs would have their ACK/NACK subcarriers encoded. As per the previous schema ideas, the first symbol being ACK and second being NACK is depicted by $ack_value[0]$ and $ack_value[1]$ respectively, as discussed previously the encoding and transmission of these fields are mutually exclusive, meaning it would be contradictory for a UE to encode both ACK and NACK as a feedback message. The arrays in \cref{fig:encodeACKSNACKS} depict the choice of ACK and NACK for the 10 UEs in the test. 

The next step is how to physically encode these values in the the PUCCH. Configurations for format 1A are still chosen as the path through the source code. Some modifications are then made to the how these bits are encoded into the UCI. This can be viewed in \cref{fig:uci_mod_tbma}

\begin{figure}[H]
    \centering
    \includegraphics[
      width=7cm,
     height=5cm,
    keepaspectratio,]{UCI_encode.png}
    \caption{UCI encoding for TBMA}
    \label{fig:uci_mod_tbma}
\end{figure}

Post the encoding of the UCI bits to the PUCCH, the rest of the procedure for creating the raw symbols to be transmitted is as normal. The next step in the proof of concept is deal with all the different symbols from the different users who would be transmitting the TBMA over the air. In the real world these symbols would be summed over the air by the fact they had been encoded in the same place on the actual radio link, as there is no actual radio link in this concept, the idea was adopted to sum these symbols up before they were processed on the eNB. \cref{fig:tbma_buf} can be seen that each of the symbols is just summed to there respective next users symbols, there are 1008 raw symbols due to the configurations. 

\begin{figure}[H]
    \centering
    \includegraphics[
      width=7cm,
     height=5cm,
    keepaspectratio,]{tbma_buf_sum.png}
    \caption{Sum of the TBMA users.}
    \label{fig:tbma_buf}
\end{figure}

At this point, we have all our symbols gathered up from multiple UE transmissions. The next step is we must process these symbols to get some useful information from them, this snipped is depicted in \cref{fig:tbma_decode}. As we are decoding two different bit fields, which potentially could have the possibility of being 4 distinct combinations of events, i.e ack/ack, nack/ack etc, however as we know the outcome of TBMA in this case is a mutually exclusive event by  each UE, i.e each UE can only transmit either an ACK or NACK and not both, we are only interested in 2 outcomes when we match the correlation. The correlation matching is taken by finding the maximum value for the 4 possibilities that exist, then taking this as the max correlation that the value is in deed the value that was sent. These two correlation values are then taken to show what the ratio of the i.e ACKs to NACKs are.

\begin{figure}[H]
    \centering
    \includegraphics[
      width=11cm,
     height=7cm,
    keepaspectratio,]{decoding_tbma_acks_nacks.png}
    \caption{Decoding of ACK/NACK message transmitted over combined channel.}
    \label{fig:tbma_decode}
\end{figure}


Although there is no real radio link in this POC, it can still be made more realistic however, by adding in a simulated random complex channel as per the simulation in the previous section. The channel, h is taken from a normal distribution with mean 0 and standard deviation 1 (for both the real and imaginary parts) and applied \ac{i.i.d} along the channel of each user. Taking a step forward to a more realistic setup as per the simulations in \cref{tbma_sec}, we add in the duplicate re-transmissions and the random channel, seen \cref{fig:retries_ota_complex}. 

\begin{figure}[H]
    \centering
    \includegraphics[
      width=9cm,
     height=7cm,
    keepaspectratio,]{retries_ota.png}
    \caption{ User Re-transmission, complex channel.}
    \label{fig:retries_ota_complex}
\end{figure}


Finally, to satisfy the power constraint on the transmission discussed also in \cref{sota}. The final buffer of symbols represented by $tbma buffer$ are then divided by the number of re transmissions which were done, this is to simulate the power of the symbols being collectively worth 1 transmission vs how they would be worth 30 transmission of power. This can be viewed below;

\begin{figure}[H]
    \centering
    \includegraphics[
      width=7cm,
     height=5cm,
    keepaspectratio,]{adapt_power.png}
    \caption{Adaptive power of transmission.}
    \label{fig:retries_ota_complex_adapt}
\end{figure}

Overall the POC has shown that the goal-orientated approach of TBMA can be work with that of the LTE network, the implementation described above can handle encoding and decoding ACK/NACK TBMA transmission with a single signature for each ACK and NACK. It follow the same procedure which would be found in a live LTE network implemented by srsLTE, with some caveats on how the encoding is performed. Obviously some deviation was needed to achieve the decoding of the channel, however there has been nothing which based on the research done in this thesis has violated any serious LTE concerns.  

 

\newpage
\section{Results and analysis}\label{sim_results}
Until now we have discussed the relevant articles to arrive at the point where a numerical simulation and link-layer simulation can be conducted and the results analysed. This section details the results, analysis and evaluation of both the numerical simulation and also the \ac{POC} in the srsLTE SDR environment. Firstly the simulation environment section will detail the numerical analysis of TBMA, comparing the effect of different users and re-transmissions for both an identical and fading channel. Next we will move onto the \ac{POC} in the SDR environment through srsLTE, compares the results of the link-layer simulation with an inclusive set of results with a modelled fading channel. 

\subsection{Simulation environment}
In this section the results from the numerical simulations which is conducted in the frequency domain, will be presented with analysis. Prior assumptions will be outlined before the the numerical simulation is presented, then a short discussion on the results comparing the assumptions with the real outcome. 

To evaluate the results of the simulation, the \ac{MSE} of the ground truth and the outcome of the TBMA was used. Some prior assumptions about the environment are also considered, where all simulations are done for an $SNR \in (-20\dots40)dB$, all users transmit their respective payload with the energy constraint  $|\textbf{s}_{k}^2| \leq 1$.  
Where the SNR and $\sigma2$ defined as:
\begin{equation} 
\text{SNR}_{\text{dB}} = 10^{\frac{SNR_{i}}{10}}    
\end{equation}
Where each SNR value is calculated at each individual value in the above range. The power of each users signal is then transmitted with the following:
\begin{align}
    signal_{power} = \frac{\sum\limits_{k=1}^{K}|S_{k}|^2 }{S_{K}} \     
\end{align}
and the White Gaussian complex noise is modelled as:
\begin{align}
    noise \approx \mathcal{CN}(0,1)
\end{align}
Finally we arrive at the output of the channel, which is the same equation as described \cref{eq:system_model}. Lastly, we have the performance metrics MSE, which is a defined as the MSE between the ground truth values and the received signal.

The following simulations were conducted in two different modes. All simulations consider random white complex Gaussian noise, $w$. First we take a identical channel response, $h=1$, which would show to us the behaviour of the system model in the best case channel scenario. Next, we consider a more realistic scenario when the channel is modelled as a response of size 3, which is modelled as a white complex Gaussian (fading channel). The comparison of the varying affect of the number of users, $k$ and re-transmissions $l$ is considered and evaluated. 
The numerical comparison of TBMA with varying;
\begin{itemize}
    \item number of users, $K$
    \item re-transmissions, $l$ with power constraint.    
    \item re-transmissions, $l$ without power constraint.
\end{itemize}

For all methods analysed numerically, the distribution is uniform, where, the probability of each subcarrier being encoded is $\frac{1}{R}$, where $R$ is the total number of subcarriers available, which coincides with the total number of events available for a given transmission, see \cref{sota}. 
  
\subsubsection{Identical Channel}

% users
\textbf{Users} - First looking at how increasing the number of users in a given simulation effects the \ac{MSE}, we would expect that in the identical channel that as the number of users increase that the  \ac{MSE} should also decrease too. Due to the identical channel we only have the additional $w$, noise value which would effect the outcome. Thus as we increase the number of users, we would reduce the effect of the additive noise. Which makes sense as when we are only adding a single noise value to all subcarriers. Therefore as the number of users increase, the probability of each event occurring also increases. This means that the noise value when all the subcarriers are normalised , should have less of an effect on the final outcome.  We would expect to see a constant decrease in the MSE as we reach higher SNR levels. 


\begin{figure}[H]
    \centering
    \includegraphics[
      width=12cm,
     height=14cm,
    keepaspectratio,]
    {users_change.png}
    \caption{Varying effect of the SNR vs MSE of different number of users in the network. }
    \label{fig:effect_of_users}
\end{figure}

In \cref{fig:effect_of_users}, it can be seen that as we increase the number of users in the network, that the previous assumption of decreasing MSE would hold still. We can also see from Figure \ref{fig:effect_of_users} that the affect of increasing the users is beginning to saturate, slowing as we increase the number of users. This observation is also a sensible assumption, that as the number of users tends toward infinity, we would see more of a true likeness of the original distribution. 

% retransmissions
Re-transmissions - Looking at the first of the two scenarios for re-transmissions, without power adaption. Without power adaption does not take into consideration the power constraint described in \cref{sota}. It would be assumed that the effect of the increasing number of re transmissions would decrease the MSE, due to the averaging effect of re-transmitting the same information reducing the error in the estimate to the real distribution.

\begin{figure}[H]
    \centering
    \includegraphics[
      width=12cm,
     height=14cm,
    keepaspectratio,]
    {retries_identical.png}
    \caption{Varying effect of the SNR vs MSE of different number of re-transmissions of the same channel.}    \label{fig:effect_of_retx}
\end{figure}

As per the previous assumptions, we can clearly see in Figure \ref{fig:effect_of_retx} increasing effect of the re-transmissions does indeed reduce the MSE of the system. This is due to the channel being deterministic, the effect of the re-transmission has less importance, as the noise $w$ is added after the summed signal at the fusion centre, thus there is nothing to average out from the received signal. This approach without taking into consideration this power constraint, would be a lot of overhead with respect to the amount of power needed due to the increasing number of re-transmissions.

Adaptive Power Re-transmissions - Lastly, looking at how the adapting the power across re transmissions affects the over all MSE. The idea is that such a system of TBMA should be constrained by power, it would be inefficient if every re-transmission and subcarrier index of the same signal was sent at full power, this would be a huge overhead for an information centric approach that utilises re-transmissions to increase the accuracy. Thus the idea was adopted such that each user would transmit their respective  re-transmission signal with power,$\frac{1}{L}$, where L is the total number of re-transmissions of the scheme. With an identical channel it would be an appropriate assumption that all the users perform the same. 
If the channel is identical, and the energy across the entire transmission $L$, of a given user $k$ is equal to $1E$ then, it woe could view this is a just a $single$ transmission as the re-transmission are not subject to any noise, but just normalised in values across the entire subcarrier range. 

\begin{figure}[H]
    \centering
    \includegraphics[
      width=12cm,
     height=14cm,
    keepaspectratio,]
    % {event_subcarriers_change.png}
    {Adaptive_power_deterministic.png}
    \caption{Varying effect of the SNR vs MSE of having an adaptive power of each signal transmitted based on the number of retransmission.  }
    \label{fig:adapt_det_retx}
\end{figure}

Looking at \cref{fig:adapt_det_retx}, we can see a downward trend in the MSE. As the assumption was before, that the each increasing of re-transmissions would appear all the same due (they are not exactly the same, however we can see from \cref{fig:adapt_det_retx} that they do not vary greatly) to the identical channel and summed normalised re-transmission.  


\subsubsection{Fading Channel}
Now, looking at the same scenarios, but using a random Gaussian complex channel. The same preconceived ideas should apply for all three use cases, however due to the more realistic channel, we should see more noisy representations of what the deterministic channel showed.

We begin the fading channel part again with the varying amount of users, we would again expect that as we increase the number of devices/users that we should also see a decrease in the overall MSE. However, unlike the identical channel, we would expect not see a constant decrease in the MSE as we move to higher SNR ranges. This would be due the effect of the signal being less noisy at higher SNR.
% users
\begin{figure}[H]
    \centering
    \includegraphics[
      width=12cm,
     height=14cm,
    keepaspectratio,]
    {users_randoms.png}
    \caption{Varying effect of the SNR vs MSE of different number of users in the network - Fading channel. }
    \label{fig:effect_of_users_random}
\end{figure}

In Figure \ref{fig:effect_of_users_random}, it can be seen that even with a random complex channel that increasing the number of users still decreases the MSE. Of course, the MSE is generally much higher in the random channel. Thus we observe the initial MSE of the corresponding plots to be different. We would also notice that the decrease in the MSE does begin to saturate at higher SNR ranges. Which could be explained by the previous assumption that the higher SNR ranges ranges, the channel is playing less of an effect on the overall performance and that at a certain point we reach the limit of the system. It is worth noting that with only 1 device/user, that the MSE unlike multiple devices seems to continue to have a lower MSE. Which could be due to the fact that one device at higher SNRs is only having its subcarriers effected by the noise added to its subcarriers, as the each subcarrier is subject to the same channel response, we could see that how the normalisation in the end accentuates the active subcarriers. 

In reality we are only interested in adhering to the power constraint on transmissions. Where the total power across all the re-transmissions equates to the total power of one transmissions. We will only simulate the fading channel for the normalised re-transmissions. As we seen in the \cref{fig:adapt_det_retx}, that due to the identical channel we observe more or less identical MSE across varying re-transmission. it would be sensible to assume that due to the fading channel we would not observe similar results. Instead, as discussed in the \cref{sota}, we would still expect that in a fading channel as we increase the number of re-transmissions that we should have better performance. 
% retransmissions
\begin{figure}[h]
    \centering
    \includegraphics[
      width=12cm,
     height=14cm,
    keepaspectratio,]
    {fading_retries.png}
    \caption{Varying effect of the SNR vs MSE of different number of re-transmissions, normalized rtx.}    \label{fig:effect_of_retx_random}
\end{figure}

We see that another main assumption that the MSE will overall decrease with more re-transmissions holds true. However, we do see some interesting behaviour while we increase the re-transmissions and move across the SNR ranges, such that different amounts of re-transmissions seem to perform better at different SNR ranges. For example, at an SNR $\approx 15dB$, we see that 16 re-transmission has a much lower MSE than 256. Which could be an interesting indicator when utilising TBMA to choose which amount of re-transmissions to choose when giving feedback. 

\subsection{srsLTE environment}
The evaluation and testing of the above proposed proof of concept method and implementation on the link layer is evaluated and tested in this section Here we take into consideration the two different approaches as per the simulation in previous sections. Initially we check the viability of the solution using a channel equal to 1, meaning an identical channel. Next, we move forward to verifying the approach with a random normal complex channel, with mean 0 and standard deviation 1. These two approaches were conducted for various random ACK/NACK combinations and varying UEs with adaptive power and fixed. For both approaches we have not considered the noise term $w$, as in \cref{eq:system_model}.

\subsubsection{Identical Channel}
Several different tests were conducted with a identical channel to see if the ratio of ACKs to NACKs could be understood. This would be a great additional feature to the LTE, as for the reasons discussed earlier about redundancy and the more effective usage of the radio resources available for larger scale networks and users. Simulations of 10, 50 and 100 users were conducted, with a range of different of parameters for the retries and the various ACK/NACK combinations. This subsection only shows an overview as the channel is identical. This identical channel is used as a bench mark to ensure the encoding and decoding of the combined TBMA channel performs as expected. 

\begin{table}[H]
    \centering
 \begin{tabular}{||c c c c||} 
 \hline
 No. UEs & No. ACK/NACK & No. Retx & Result \\ [0.5ex] 
 \hline\hline
 10 & 0.3/0.7 & 1 &  0.3/0.7 \\ 
 \hline
 50 & 0.3/0.7 & 1 & 0.3/0.7 \\
 \hline
 100 & 0.6/0.4 & 1 & 0.6/0.4 \\
 \hline
 200 & 0.6/0.4 & 1 & 0.6/0.4 \\
 \hline
 50 & 0.3/0.7 & 1 &  0.3/0.7 \\ 
 \hline
 50 & 0.3/0.7 & 8 & 0.3/0.7 \\
 \hline
 50 & 0.6/0.4 & 16 & 0.6/0.4 \\
 \hline
 50 & 0.6/0.4 & 32 & 0.6/0.4 \\ [1ex] 
 \hline
\end{tabular}
    \caption{Results of deterministic channel, increasing UEs srsLTE POC}
    \label{tab:identical_ue_srs}
\end{table}

As we can clearly see from \cref{tab:identical_ue_srs}, we get perfect results. This is expected as we introduce no noise or fading channel. However, this information is still valuable because we can see that the summed symbols during the encoding and decoding process are still able to be understood by the LTE stack. This allows us to understand that the proposed approach is at least viable from an encoding and decoding point of view. 

\subsubsection{Fading Channel}
The same setup and tests as above for the a random complex channel were also conducted, this random complex channel is modelled as $\mathcal{CN}(0,1)$. However, now as this channel is no longer deterministic, the below results are more specific and to the individual simulations conducted in the previous sections of this thesis. We will look at the results of the following situations, (all with normalised re transmissions):

\begin{enumerate}
    \item Increasing UEs, fixed re-transmissions, fixed ACK/NACK. 
    \item Increasing re-transmissions, fixed UEs, fixed ACK/NACK
    \item Varying ACK/NACK, fixed UEs, fixed re-transmissions.
\end{enumerate}

Situation 1, we would expect that the increasing number of UEs in the feedback TBMA , that the received and decoded ACK/NACKs would tend towards the real world values that had been transmitted. 

\begin{table}[H]
    \centering
 \begin{tabular}{||c c c c||} 
 \hline
 No. UEs & No. ACK/NACK & No. Retx & Result \\ [0.5ex] 
 \hline\hline
 10 & 0.4/0.6 & 1 &  0.086/0.913 \\ 
 \hline
 50 & 0.4/0.6 & 1 & 0.218/0.782 \\
 \hline
 100 & 0.4/0.6 & 1 & 0.288/0.712 \\
 \hline
 200 & 0.4/0.6 & 1 & 0.346/0.654 \\ [1ex] 
 \hline
\end{tabular}
    \caption{Results of random channel, Increasing UEs srsLTE POC}
    \label{tab:rand_chan_UEs_increase}
\end{table}

As expected we see in \cref{tab:rand_chan_UEs_increase} that as we increase the number of UEs we move closer to the real \ac{PMF} of the transmitted data. 
Next we look at the varying affect of the normalised re-transmissions. We would expect that as we increase the re-transmissions that we would see a more likeness of the distribution of the original data.

\begin{table}[H]
    \centering
 \begin{tabular}{||c c c c||} 
 \hline
 No. UEs & No. ACK/NACK & No. Retx & Result \\ [0.5ex] 
 \hline\hline
 50 &  0.4/0.6 & 1 &  0.218/0.782  \\ 
 \hline
 50 &  0.4/0.6 & 2 & 0.268/0.732 \\
 \hline
 50 &  0.4/0.6 & 8 & 0.312/0.688 \\
 \hline
 50 &  0.4/0.6 & 16 & 0.306/0.694 \\
 \hline
 50 &  0.4/0.6 & 32 & 0.342/0.658 \\ [1ex] 
 \hline
\end{tabular}
    \caption{Results of random channel, Increasing re-transmissions srsLTE POC}
    \label{tab:rand_chan_rtx_increase}
\end{table}

Which again, we can see as we increase the number of re-transmissions we indeed move closer to the original distribution. Finally for completeness, we look at the performance of the varying ACK/NACK combinations, we would assume that the combination of ACK/NACK being transmitted shouldn't make much of a difference, perhaps when All the UEs have a equal consensus on the outcome we would see slightly undesirable behaviour and when things are equal we may not 

\begin{table}[H]
    \centering
 \begin{tabular}{||c c c c||} 
 \hline
 No. UEs & No. ACK/NACK & No. Retx & Result \\ [0.5ex] 
 \hline\hline
 50 & 0.8/0.2 & 32 &  0.891/0.109 \\ 
 \hline
 50 & 0.1/0.9 & 32 & 0.02/0.98 \\
 \hline
 50 & 0.5/0.5 & 32 & 0.61/0.39 \\
 \hline
 \hline
 50 & 0.0/1.0 & 32 & 0.0/1.0 \\
 \hline
 50 & 1.0/0.0 & 32 & 1.0/0.0 \\ [1ex] 
 \hline
\end{tabular}
    \caption{Results of random channel, varying ACK/NACKs srsLTE POC}
    \label{tab:rand_chan_acks_change}
\end{table}

As we can see in \cref{tab:rand_chan_acks_change}, the results for the first three elements in the table are far from perfect, but we can see that they would give us a likeness of what is happening. Which in the end, is idea for the likes of our use-case. The last two results are rather interesting as they are actually perfect even with this channel. Which could be down to that fact that there is no additional noise, $w$ term added after the sum of the symbols, this coupled with the fact that the particular subcarriers which would house the NACK is 0, thus 0 times the fading channel is still 0. 

If we compare \ac{TBMA} to normal unicast feedback, we can see that of course unicast would indeed be more accurate, knowing exactly how many ACKs or NACKs. However the number of UEs which an \ac{eNB} can serve is limited with this approach. On the other hand, with TBMA we would be able to serve multitudes more UEs but with a decrease in the accuracy. Which as discussed, may not be an issue for the intended application. Overall we can see that the both increasing the number of User and re-transmissions play important roles in the increasing the accuracy of the approach.

\newpage
\section{Summary and outlook}\label{summary_out}
Arriving at the end of this thesis, we have presented and discussed \ac{TBMA} from a generic numerical analysis point of view and also in terms of the LTE framework with a link-layer analysis. This thesis has presented a goal orientated approach for multiple users to transmit information across a wireless medium using type based multiple access. Firstly, in a simulation environment, looking into which attributes of the simulation play the biggest role into minimising the error. Then walking through the stages of how a possible implementation could look in the LTE framework, to a possible \ac{POC} in an LTE \ac{SDR} project at link-layer. Now we would move onto summarise this thesis and talk about the potential outlook for future work. 

\subsection{Summary}
We can see that the number of re-transmissions of the same data helped increase performance of the transmission regarding moving toward the true \ac{PMF} of the transmitted data. However, this led to an interesting idea that all nodes in a multicast group sending re-transmitted data at full power is both redundant and counter productive, so having a power constraint that all transmissions for a TBMA channel, should not exceed the combined power limit of 1E and when multiple subcarriers are being used, the power of any encoded subcarriers combined should also not exceed this limit.

 
Looking into how to apply this TBMA to real world LTE framework was the next step, looking closely how the like of ACK/NACK re transmission works in the current setup, was the introductions to knowing how TBMA in relation to ACK/NACK could be integrated into the likes of the current LTE network. Identifying how the simplest form of feedback worked in LTE led to a design choice around the PUCCH transmission of ACK/NACK through format 1A. Further investigation into how the the  mapping of these high level concepts of ACK/NACK to the physical resource was investigated. 

Upon understanding how these feedback channels works through 3GPP specification led to a proof of concept implementation through a LTE complaint open source SDR project called srsLTE. This project adheres to the same implementation found in the technical specifications from the 3GPP. Through the use of some examples which follows the normal LTE procedure for encoding and decoding up-link transmission. An adaption of the test was created to allow for the encoding and decoding of a 2 subcarrier TBMA channel with ACK/NACK in mind. Although this implementation was not strictly done over a radio interface, it was encoded/decoded as in normal LTE mode through the srsLTE project. In an attempted to ensure the most realistic representation of the setup without an actual radio interface, a modelled complex channel was introduced to bridge this gap. 


\subsection{Outlook}
Moving forward and next steps, would be to trial the same system over a physical radio link. Although the simulated channel is a good model of the real world, to ensure this would be a completely viable solution, it should be trialled with a real radio link. Upon verifying that this assumption is correct, next steps should be to implement the other encoding schemes as per the \cref{schemas_and_channels}. 

A step further which could yield more stable and realistic results for underlying PMF could be each user transmitting a recorded PMF of their respective \ac{qoi} over time. This approach might be more realistic than its counterpart in the previous paragraph, due to the fluctuating nature of real world events in an outdoor environment. 

With the eNB having a more generalist snapshot of e.g CQI overtime vs an instantaneous approach could avoid unnecessary modulation coding changes when they are not needed. Such that if a user $k$, transmits a PMF of CQI values, the total number of of energy must not exceed $1E$. For example, user $k$, observes \ac{CQI} values of $8,9,10,11$. The constraint is such that in the observation period  4 values have been observed. Thus $k$ must share the total energy of 1 transmission with across those values, which in this case would mean each subcarrier receiving $\frac{1}{4} \times E$.

With regards to the link layer implementation, an extension from 2 fields for ACK/NACK to multiple fields capable of conveying more granular data would be a suitable next step, to facilitate a transmitted CQI situation. The current setup for encoding ACK/NACK would also need to be extended for encoding PMFs from the UEs, currently this POC only encompasses a one-shot style on one signature on the PUCCH. 

This thesis has only investigated how the TBMA approach could be integrated into the lower layers of the LTE network, more investigation into how the higher layers of the LTE stack would play a role in the formation and organisation of these TBMA packets. More over, other considerations to be taken into account moving forward would be such items as, testing the viability of integrating this approach with live ACK/NACK date from the unicast channel, how could this be multiplexed given the current setup in LTE. Other questions remain open too - should this TBMA have its own predefined format for the PUCCH or should it just utilise the remaining free areas in the PUCCH/PUSCH, what style of priority should be used. This proof of concept implementation merely address the physical how to of TBMA and does not take into consideration other related questions about how the finer integration should work.


Overall as expected the result for the deterministic approach were always perfect representations of the ground truth. Although this is not a realistic situation, it has given the basic backbone for knowing that the process of encoding and decoding the TBMA channel was indeed plausible in the context of the LTE framework. Contrast to this, looking at the result when there was a complex random channel introduced between the encode/decode of the symbols, we can see that although exact ratios were not achieved. However, we see that the resultant ratios are still of within suitable range of error, and could be of use in the real world. 

With regards to the actual viability of \ac{TBMA} for usages as a feedback method for multicast channels we can see that how technically the TBMA approach could theoretical facilitate millions of users at once on a single \ac{eNB}. When we compare this with the likes of the example presented in \cref{sota}. We can see some pretty obvious limitations, such as number of transport blocks and periodicity of feedback that severely limit the number of devices an \ac{eNB} can serve. When we couple this with the unprecedented growth in the number of connected devices, we begin to see some serious issues with current methodologies ability to scale to meet demand. 

\newpage
\section{Glossary}\label{glossary}


% %% here i talk about how to estimate the parameter or pmf of the transmitted signal, again i split this into both h=1 and random.
% The idea of a TBMA  approach is to best design the channel so that the error of the estimated empirical distribution can be minimised, such that it tends toward 0, the general function $D$ has been defined in \cref{sota} and as discussed can take the form of different metrics as a definition of distance. In the case of this thesis we are interested in the underlying PMF, $p_{\theta}$ of the data transmitted. We access the approach by 
% % \begin{align}
% % {Q}^* = \min_{\hat{Q}} D(\hat{p}\|p_{\theta}),  
% % \end{align}

% % Where the min $ \min_{\hat{\theta}} D$ and $ \min_{\hat{Q}} D$ are measured using the Mean squared error, which is defined as:
% % % lips has the figure on the first page!! describing the appplication of cqi feedback. it has the use case of cqi describeed there already. 
% % \begin{figure}[H]
% %     \centering
% % \begin{equation}
% %   MSE = \frac{\sum\limits_{S=1}^{S}|\theta -\hat{\theta}|^2 }{S} \
% % \end{equation}
% %     \caption{Mean Square Error between Ground truth and outcome.}
% %     \label{fig:MSE}
% % \end{figure}

% Where, $Q*$ is PMF of the received signal which exist in the defined probability space $[R]$. We assume that there is no prior knowledge of the distribution $p_{\theta}$, since the idea of the approach is to derive an estimate of the distribution of the data. 
\begin{align}
    L(\theta) = \prod_{i=1}^n\frac{2x_{i}}{\theta^2}e^{-\frac{x_{i}^2}{\theta^2}}
\end{align}
\begin{align}
    L(\theta) = \prod_{i=1}^n x_{i}(\frac{2}{\theta^2})^n e^{-\frac{\sum{x^2}}{\theta^2}}
\end{align}

Then taking the $ln$ log of both sides to simplify:

\begin{align}
    \ln{L(\theta)}= \ln{(x_{1},x_{2}\dots,x_{n})} + n\ln{2} - n\ln\theta^2 {-\frac{\sum{x^2}}{\theta^2}}\ln{e}
\end{align}
\begin{align}
    \ln{L(\theta)}= \sum\ln{x_{i}} \cdot + n\ln{2} - 2n\ln{\theta} -\frac{\sum{x^2}}{\theta^2}
\end{align}

Differentiating with respect to $\theta$, gives:

\begin{align}
    \frac{\partial }{\partial \theta}\ln L{(\theta)} = -2n\frac{\partial}{\partial\theta}\ln\theta - \sum x_{i}^2\frac{\partial}{\partial\theta}(\frac{1}{\theta^2})
\end{align}

\begin{align}
    \frac{\partial }{\partial \theta}\ln L{(\theta)} = -\frac{2n}{\theta} + \frac{2\sum x_{i}^2}{\theta^3} \dots \dots \dots
\end{align}

Now setting the partial derivative of $\theta$ equal to $0$ so we can get a maximum.

\begin{align}
    \frac{\partial }{\partial \theta}\ln L{(\theta)} = 0
\end{align}

\begin{align}
    -\frac{2n}{\theta} + \frac{2\sum x_{i}^2}{\theta^3} = 0
\end{align}
\begin{align}
    \frac{n}{\theta} = \frac{2\sum x_{i}^2}{\theta^3}
\end{align}
\begin{align}
  n\theta^2= \sum x_{i}^2
\end{align}

Thus we attain:

\begin{align}
    \theta^2 = \frac{\sum x_{i}^2}{n}
\end{align}


\addcontentsline{toc}{section}{References}
\bibliographystyle{plain}
\bibliography{\jobname}
\begin{thebibliography}{9}
\bibitem{3gpp36321}
3GPP TS 36.321 Medium Access Control (MAC) protocol specification, Release 15, 2020
\bibitem{3gpp38321}
3GPP TS 38.321 Medium Access Control (MAC) protocol specification, Release 15, 2020
\bibitem{3gpp25319}
3GPP TS 25.319 Enhanced uplink; Overall description, Release 13, 2016
\bibitem{ETSITS136213}
Evolved Universal Terrestrial Radio Access (E-UTRA), Physical layer procedures 
\bibitem{multiple_access_protocols}
Multiple Access Protocols, Performance and Analyisis, Raphael Rom, Moshe Sidi 
\bibitem{shannon_theory}
A Mathematical Theory of Communication, By C. E. SHANNON
\bibitem{information_centric}
Din, I.U., Asmat, H. and Guizani, M. A review of information centric network-based internet of things: communication architectures, design issues, and research opportunities. Multimed Tools Appl 78, 30241–30256 (2019). 
\bibitem{lte_advaned_mobile}
4G: LTE/LTE-Advanced for Mobile Broadband, Erik Dahlman, Stefan Parkvall, Johan Skold
\bibitem{tbma}
G. Mergen and L. Tong, "Type based Estimation over Multiaccess Channels" in IEEE 
\bibitem{access_tech}
Chen, W.K., "The Electrical Engineering Handbook", P.1005, Section 7.1 Access Technologies, 2004.
\bibitem{mod_impact}
Impact of Modulation Schemes on LTE, Juhi Pruthi, Pooja Agarwal, Nikita Jain.
\bibitem{umts_sesia}
LTE – The UMTS Long Term Evolution: From Theory to Practice Stefania Sesia, Issam Toufik and Matthew Baker.
\bibitem{csi_defs}
R\&STS8980 test system analyzes LTE quality indicators: CQI, PMI and RI
WIRELESS TECHNOLOGIES | Conformance test systems. 
\bibitem{fdma_info}
Chen, W.K., "The Electrical Engineering Handbook", P.1005, Section 7.1.1 FDMA, 2004. 
\bibitem{tdma_info}
Chen, W.K., "The Electrical Engineering Handbook", P.1006, Section 7.1.2 TDMA, 2004.
\bibitem{cdma_info}
Chen, W.K., "The Electrical Engineering Handbook", P.1007, Section 7.1.3 CDMA, 2004.
\bibitem{ofdma_info}
Sassan Ahmadi, LTE-Advanced,2014. Orthogonal Frequency Division Multiple Access. Section 9.5
\bibitem{srsLTE}
srsLTE opensource project, https://www.srslte.com/ 
\bibitem{los_pic}
Line of Sight Infographic, https://unlimitedlteadvanced.com/4g-lte/how-do-i-choose-a-4g-lte-antenna/
\bibitem{interference_pic}
http://hydrogen.physik.uni-wuppertal.de/hyperphysics/hyperphysics/hbase/sound/interf.html
\end{thebibliography}
\end{document}

